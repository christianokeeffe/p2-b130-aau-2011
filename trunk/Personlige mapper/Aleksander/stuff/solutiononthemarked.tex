\section{Solutions on the market}

The amount of lists and guides on the market is huge. These lists and guides offers help and provide tips for packing for travelling. Some of these lists and guides have been developed into apps(application) that are available for the customer to use.
There also exists programs, that have integrated algorithms to handle optimization of the packing, on the marked that can be used.
First a look into these lists and guides and the more advance solution thereafter.

\subsection*{App - Packing Pro}

Packing pro is an app that offers templates for check lists to the customer. These templates are designed to different purposes regarding the customer, gender, type of trip, and purpose of the trip.
The customer can then load the wanted template for the purpose. Packing Pro does nothing else than offer a management tool that helps the customer get an overview of all the things to remember. Packing pro is an app for the smart phone and thereby does not need a running computer to use. As the name implies(pro) the app have to be bought before it can be used \citep{packingpro}.

\subsection*{App - Checkmark Packlist}

Checkmark Packlist offers different kinds of templates like Packing Pro. Between these two products there are almost no variation expect for the applications GUI(graphical user interface). This is also an app for the smart phone and thereby easy to access and use.
This app needs additional software to work and thereby slightly more annoying for the customer to use this product. The customer will also have to pay an additional price to get the full product \citep{checkpacklist}.

\subsection*{Online check/tip list}
The online check list works as a reminder when packing luggage. It also give tips and tricks that could be considered when packing for the trip. There exists a lot of different websites offering this service for free. Some are posted by an organization and others by a person on a forum.
\citep{onlinecheck} is an example of this kind of website. This website offers a list of 10 tips that can be helpful for the customer when they are packing for a trip. The site does not help with the actual packing, instead it helps with the planing of materials that the user might want to have on the trip.

\subsection*{The e-Commerce shipping calculator}

The e-Commerce shipping calculator is an advanced program that helps the customer packing large containers and calculates the price of the shipment.
By typing the size, weight, location, and destination of the items that should be shipped, the program can calculate what the prize is going to be and generates a 3D(3 dimensional) model of the container where the given items are placed in the best possible way so there are a minimum of wasted space. On their website \citep{solvingmaze} they offer a demo(demonstration) of their program.

\subsection*{Solutions coverages of the requirements}

In this section there will be looked into what the different existing programs features and compare them to the requirements that have been set to solve the given problem. The comparison will be in the form of a table that gives an overview of the coverage of the requirements. The table will be used to conclude if there is already a program that solves this problem.

\begin{table}[H]
\begin{center}
\begin{tabular}{c  c | c | c | c | c | c}
\textbf{Included in product} &  \rotatebox{90}{\textbf{Solutions}} &\rotatebox{90}{App - Packing / Packing Pro} & \rotatebox{90}{App - Checkmark Packlist}& \rotatebox{90}{Online check/tip list}&\rotatebox{90}{The e-Commerce shipping calculator}\\ \hline
Guide the user & & x & x & x & x   \\ \hline
Distribute weight &  &   &   &   & x    \\ \hline
Distribute space  &  &   &   &   & x    \\ \hline
Account for the trips length  & &  x  &  x  &  x  &  x  \\ \hline
On the road   &  &  x  &  x  &   &  x  \\ \hline
Baggage rules  &  &    &    &   &   \\ \hline
Where in the suitcase  &  &   &   &   &  x   \\ \hline
Solid/liquid/bendable shapes &  &  &   &   &   \\ \hline
Packing list &  & x & x &   & x    \\ \hline

\end{tabular}
\end{center}
\caption{ Table for the different products on the market compared to the requirement seen in section \ref{sec:spec}}
\label{tab:OtherPrograms}
\end{table}

On table: \ref{tab:OtherPrograms} it can be concluded that none of the found program fully covers the requirements set to solve the problem at hand.
Therefore it can be conclude it is relevant to work with this problem and try to solve it.