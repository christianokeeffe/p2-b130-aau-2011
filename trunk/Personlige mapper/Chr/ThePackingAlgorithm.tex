\section{The Packing Algorithm}
\label{sec:algorithm}
The algorithm used for packing the item in this project is based on the known theory of the Bin Packing Problem (see section \ref{sec:binpacking}). The algorithm has been a mix of some elements from the First Fit Decreasing and the Best Fit method. These has been combined to make the algorithm used in the project. The algorithm is inspired of \citet{three-dim-pack}. The following sections will describe the used code.
The code will try to place all the items in the luggage so it is packed most optimal in both sized and weight.

\subsection{Optimization of weight}
The optimization of the weight is done, so no luggage is exeeting the weight limit if is could be distributed different. Futhermore it would be preferable for the user, that no luggage is very heavy and another very light.
To optimize the distribution of the weight, the average weight per luggage should be calculated. It is calculated as seen on equation \ref{eq:avg_weight}, where N = Number of items.

\begin{equation}
	\label{eq:avg_weight}
	\mathrm{Avarage~Weigt} = \frac{\displaystyle\sum_{i=1}^{N} I_{weight}}{N}
\end{equation}

It is possible to distribute the weight average in the suitcases, when the optimal weight for each suitcase (the average weight calculated by equation \ref{eq:avg_weight}) is known. The program will try to distribute the weight equally, but not if it will mean that the luggage cannot be packed. Therefore the weight distribution is a optimization goal, but not as important as the volume of the luggage. This part of the optimization is a Best Fit, because it find the best luggage for an optimal weight distribution.

\subsection{Optimization by size}
The algorithm use the First Fit Decreasing (FFD) method to pack the items in the luggage. This means, that the algorithm at first sorts the items by size. It will start by packing the largest items first, which will give a better result for the packing. When the list is sorted, it will then also sort the list of luggages by size. The algorithm can now go to the actual packing. 

The general algorithm is:
\begin{itemize}
	\item Sort the items by size
	\item Check if the item can be in the first luggage, while the luggages total weight does not exit the average weight per luggage
	\item	Check if the item can fit in the luggage, else check the next
	\item  If the item cannot be fitted, exclude it from the list, and notify the user
\end{itemize}

This is a very general overview of the algorithm, and does not explain the process of the algorithm detailed enough. To explain the algorithm, a flowchart can be seen on figure \fxfatal{indsat figur}.

As seen on the flowchart, the first process which is done, is to sort the suitcases and items. They are sorted by size, so the biggest items are stored in the biggest suitcases. This will ensure the most effective way of packing, because it will fill the items, which is the hardest to fit first. The next step for this algorithm is to call the function, which will calculate the average weight per suitcase for an even weight distribution. The algorithm can now start the actual packing of the 