\section{Luggage rules}
This section will focus on the general rules regarding luggage when going abroad, whether by plane, train or other.

\subsection*{Luggage table}
This table displays the varies rules of luggage on public transport
\begin{table}[H]
\begin{center}
\begin{tabular}{| c | c | c |}
\hline
Type of luggage &  Dimension limit & Weight limits \\ \hline
Check-in luggage(Airplane) & 158 cm (Height+Width+Depth ) & 20-23 kg \\ \hline
Carry on(Airplane) & Height 50-55 cm + Width 40 cm + Depth 18-25 cm & 5-8 kg \\ \hline
Extra luggage(Airplane) & 158-277 cm (Height+Width+Depth) &  20-45 kg \\ \hline
Luggage(Train) & 100 x 60 x 30 cm & Within reason \\ \hline
\end{tabular}
\end{center}
\caption{Carry on luggage is the only one with restriction on the specific dimensions. \citep{General_rules}}
\end{table}

\subsection{Charter trips on air planes}
Given below is the rules for varies items you could bring on the plane
\newline
\newline
Checked-in luggage
\newline
Items not allowed:
\newline
Explosives, corrosive and flammable compounds e.g. gas, methylated spirit, paint and the like  
\newline
Oxygenated, toxic and radioactive compounds 
\newline
Flammable gases 
\newline
Magnetic materials 
\newline
Fireworks 
\newline
Sedatives 
\newline
Bleach 
\newline
Paint
\newline\newline
Carry on
\newline
Approved items:
\newline
Liquids, perfume, gel and spray – max. 100 ml – equal to one decilitre pr. container \newline
You are only allow to bring these containers (bottles, cans, tubes and, so on), if they are contained in a transparent plastic bag, which have to be closed (1 litre bag per passenger).
The bag have to be resealable.
\newline
Past security, wares can be purchased (including spirits, perfume and other liquids). Wares are handed out in sealed bags, these bags may only be opened after the final destination have been reached.
\newline
It is now a requirement that you take off your overcoat, take laptops and other larger electronic devices out of the bag before the security check-in.\\
\citep{Prohibited_luggage}

\subsection{Rules on trains}

There are different rules depending on which train company you are using. 
\newline
The Danish train company, DSB, have very few rules regarding the luggage you are allowed to bring with you. 
\newline\newline
The only other rule is that your luggage need to be able to lie on the luggage rack or under the seat and not be bothering or putting any other person on the train in danger \citep{rulestrain}.
\newline\newline
Another example could be Indian Railways where the luggage is allowed to have different weight depending on which class you are on. They have no other rules regarding luggage \citep{idianrules}.