\section{Luggage allowance}

Due to the hijacking and crashing of the airplanes into the World Trade Center on the 9th of September 2001, the security of airports have increased dramatically. Some of the hijackers carried knives and box cutters and this led to an immediate restriction of any and all types of sharp objects. The reason the hijackers could get these weapons on board the plane, was lax security around for instance Swiss army knives and blades like a box cutter. Along with stricter rules for items allowed on the plane, a thorough check up of the security personnel hired by the airport have been issued. After the change, airports are no longer allowed to hire their own security personnel due to a lack of discipline and training and in some cases hiring of personnel with a criminal background.\citep{Stricter_rules}\\
\\
On the 5th of October 2006 more regulations were introduced to prevent passengers from bringing liquids of too large a quantity on board (see section \ref{sec:LugRules}). To construct a bomb a certain amount of "liquid" is required for it to have enough power to be a threat, and studies have showed that several 100 milliliter containers stored in a 1 liter bag equals around 500 milliliters of liquids which in turn is not enough to make a bomb that can take down a plane. This restriction covers all types of liquids because the screening points at the security can not distinguish one liquid from another without the security personnel manually checking the various liquids, which would severely slowdown the whole process.\citep{Why_rules}\newline

Due to these restrictions packing a bag is not as simple as it used to be. A lot of items are no longer allowed and thus it can be difficult to know what is allowed and what is not. As the restriction covers all sorts of liquids packing a simple toilet bag is time consuming.
