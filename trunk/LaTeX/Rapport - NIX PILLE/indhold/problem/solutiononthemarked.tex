\section{Solutions on the Market}
\label{sec:Solution}
This section is used to research the market of packing program and thereby get an image of which solutions there already exists on the market. Through the research it is also possible to determine, how the existing solutions help the users with the stated problem. Looking at existing solutions can be used to determine which features that would be needed in the program for this project, and from these solutions get inspiration.
The amount of lists and guides on the market is huge. These lists and guides offers help and provide tips for packing when traveling. Some of these lists and guides have been developed into solutions that are available for the customer to use.

There also exists programs, that have integrated algorithms to handle optimization of the packing, on the market that can be used.
%First a look into these lists and guides and the more advance solution thereafter.

\subsection{App - Packing Pro}

Packing pro is an app developed for the Iphone and Ipad that offers templates for check lists to the customer. Packing Pro uses a touch interface which means that the user by using the finger can navigate around.

\figur{0.8}{Iphone.JPG}{This is a picture of 2 menus from the Packing Pro app \citep{packingpro}.}{fig:ipack}

Packing pro is designed with a panel in the bottom of the screen that allows the user navigate through the menus. Packing Pro provides the user with a help menu that contain information on how to use the app. On Figure \ref{fig:ipack} an example of how a check list could look like be seen.

These templates are designed for different purposes regarding the customers gender, type of trip, and purpose of the trip.
The customer can then load the wanted template for the given purpose. The user also has the possibility to create their own lists by adding things that should be remembered for the trip by themselves, The user can select an existing list and delete the objects that were found irrelevant by the user. The user can then check objects on the lists off as it gets packed. Packing Pro is a management tool that helps the customer get an overview of all the things to remember. As the name implies(pro) the app has to be bought before it can be used  \citep{packingpro}.

Packing Pro works as a checklist and helps the user remember what to pack, but it does not preform any packing of luggages content itself. So Packing Pro itself does not solve several of the found problems, for example people packing to overweight, but helps the user remember what to pack. So a feature to consider from this program is the check list function that gives the user an overview of things to pack.
A function that would not be needed is the compatible with iPhone/iPad operating system. It would be nice if the program made works cross platforms but it is not required to solve the problem.

%\subsection{App - Checkmark Packlist}

%Checkmark Packlist is an free app for the smart phones running the Android system.
%Checkmark Packlist offers different templates for check lists that the customer can use. One of these templates is the list for packing for a trip. That way the customer can select and use this template for remembering what they will need to pack for the trip. Checkmark Packlist uses touch to navigated in the program. This means that the customer with their fingers can navigated through the check list and check off things that have been packed.

%\figur{0.4}{checkmark.JPG}{Picture of the Checkmark Packlist in action from \citep{checkpacklist}}{fig:checkmark}

%On figure: \ref{fig:checkmark} there can be seen an example of the product and how Pheckmark Packlist looks like for the user.
%Checkmark Packlist does not provide customization tool that let the customer add more categories to the check list. This is only a featured provided in the paid version of Checkmark Packlist \citep{checkpacklist}.

%This app does not provide a solid solution for luggage packing but helps the user remembering what there should be packed for trip. This check list feature give a guiding effect and this is a useful feature and can be used in the product design to solve the problem. A feature to consider is the mobility by designing the program for hand held devices.

\subsection{Online check/tip list}

The online check list works as a reminder when packing luggage. It also gives tips and tricks that could be considered when packing for the trip. There exists a lot of different websites offering this service for free. Some are posted by an organization and others by a person on a forum. This means that all electronic devices as computers, tablets, and smartphones that have access to the Internet can open the website address.

An example of this kind of website is the following source \citep{onlinecheck}. This website offers a list of 10 tips that can be helpful for the customer when they are packing for a trip. The website is purely text based and helps the user packing through the tips on the website.
The site does not help with the actual pack, instead it helps with the planning of materials that the user might want to have on the trip.
The website is designed with a menu on the left that lets the users navigate through the different content on the website.

The online check/tip list in itself does not give the customer a solution to the packing problem. The websites instead helps the customer plan the trip. The type of check/tip list used on \citep{onlinecheck} does not apply as a useful feature that could be used in the final program. %Instead it would more be focus on helping the user with the packing.

\subsection{The e-Commerce shipping calculator}

The e-Commerce shipping calculator is an advanced program that helps the customer pack large containers and calculates the price of the shipment.
By typing the size, weight, location, and destination of the items that should be shipped, the program can calculate what the prize is going to be and generates a 3D-model of the container where the given items are placed in the best possible way so there are a minimum of wasted space. On their website \citep{solvingmaze} they offer a demo of their program. Their demo web browser based and thereby should be accessible from computers connected to the internet.
The demo is designed to have the containers dimensions variables and weight limit as input fields. Under the container there is a list of items where each item can have different dimensions and weights. To the right of this field is the 3D-model placed which will be generated.

\figur{0.6}{shippingcalculator.JPG}{Screenshot of the demo running taken from \citep{solvingmaze}}{fig:calculator}

The customer has to type all known data in and press "Calculate Rates" and the program will then form a 3D-model, this can be seen on Figure \ref{fig:calculator}.
This product has a number of useful features that can be used in the final product. This solution can take an item's dimensions and weight and calculate the most optimal placement in the container. This can be related to packing your luggage for a trip.

\subsection{Summing Up}

This section's main object is to look at the wanted features and recap them.
Packing Pro provide the user with a sort of check list that can be checked off and thereby help the customer remember what has not been packed yet.
Packing Pro allows the user to edit the check list which is needed if the user should be able to make a personal list.

This app does not help the user arrange the luggage content or take in consideration the size and weight of it. Thereby the app does not help people with the packing itself but rather which items that should be remembered for the trip.
The online check/tip list provide the user with advice for the trip and what to pack. Advice are great to get a general idea of what to pack but it still does not give a more effective way to pack.
The e-Commerce shipping calculator is the solution with the most wanted features. One of the strong features that can be used is the ability to calculate the most effective way a container should be packed. An important side note is that the intentions is not to pack suitcases but the feature can be related to packing contents of a suitcase.

\begin{table}[H]
\begin{center}
\begin{tabular}{c  c | c | c | c | c }
\textbf{Included in product} &  \rotatebox{90}{\textbf{Solutions}} &\rotatebox{90}{App - Packing / Packing Pro} & \rotatebox{90}{Online check/tip list}&\rotatebox{90}{The e-Commerce shipping calculator}\\ \hline
Guide the user* & & x & x & x   \\ \hline
Distribute weight &  &  &   & x    \\ \hline
Distribute space  &  &  &   & x    \\ \hline
%On the road   &  &  x  &  x  &   &  x  \\ \hline
Packing of items  &  &  &   &  x   \\ \hline
Packing list &  & x &   &     \\ \hline

\end{tabular}
\caption{ Table for the different products on the market and their features.\newline 
*The program should be able to guide the user through the different steps of the program.}
\label{tab:OtherPrograms}
\end{center}
\end{table}

Table \ref{tab:OtherPrograms} consists of features listed vertical and existing solution on the market listed horizontal. The crosses indicates when the particular product has the particular feature. The purpose of the table is to give an overview of the products and their features that were found essential to the problem.
Table \ref{tab:OtherPrograms} shows the there are a lot of help regarding what to bring, but when it comes to packing it is only one of the selected solution that had this feature.