\section{Method}
This project structure will be based on Aalborg PBL (problem based learning). The Aalborg PBL is a method whereby the learning process lies in the work with a problem and try to develop a solution for the given problem.
The Aalborg PBL method also trains the students ability to work together in a project group and give them tools to handle the processes that goes with working in a group.

The first stage of the project is the problem analysis in chapter \ref{chap:problem}, which purpose is to find and document that there is a problem to begin with. From the problem analysis a thesis statement is formed and is used to produce a list of product requirements.
The requirements are then used to design and develop a product that should solve the problem stated in the thesis statement. The design will be describe in chapter \ref{chap:design}.
The development will also have it own chapter were the program will be describe and how the different functions are made. This can be seen in chapter \ref{chap:development}.
The program are then tested on the target group of the problem. The testing phase will be described in chapter \ref{chap:testing}. The result of the testing will lead to improvements and a conclusion of the project. The conclusion will sum up the project and try to answer the thesis statement, the conclusion can be seen in chapter \ref{chap:conclusion}. This is the main course of the project, when using the Aalborg PBL model.
This project form is used because it finds and document a problem and then through the work with the problem gives an estimated solution to the problem.\\

To document the problem, a lot of information is needed. The information is found through different sources such as; books, article, websites, etc. When using information found through the internet or other sources it is important to evaluate the used sources.
This is done to filter out unreliable sources and thereby achieve a better and more trustworthy project.
This process of evaluation is also known as source criticism and are general used when using others materials as documentation. Therefore it is also a relevant method to use when using sources in the project work.\\