\section{GUI Description}

The GUI(Graphical User Interface), as the name implies, is the interface the user interacts with when operating the program.
The main interface window of the program looks like this.

\figur{0.5}{Mainwindow.png}{Main window}{fig:Main}

The main window contains the instructions on how to use the program, an "About" button which tells who the creators are and what they do. A "Manage Items" and "Manage Suitcases" button where the user can add/edit/delete items/suitcases. The button "Load Saved List" can be used if the user have previously made a list and wants to use it and/or add/edit some items or suitcases from the list. The "Start Packing" button initiates the program's algorithm and packs the suitcase(s). The progress bar is associated with the "Start Packing" button and starts when the user presses the button.

\figur{0.5}{ManageItems.png}{Manage items window}{fig:ManageItems}
The Manage Items menu is where the user can add, delete and/or edit items.

\figur{0.5}{ManageSuitcase.png}{Manage suitcase window}{fig:ManageSuitcase}
Add/delete/edit suitcases in the Manage Suitcase menu.

\figur{0.5}{About.png}{About window}{fig:About}
The About button which tells who the programmers are, and when the program was made.

\figur{0.2}{AddItem.png}{Add item window}{fig:AddItem}
When the user presses the "Add Item" button in Manage Items, this form shows. In the form there is a text box, where the user types the name of the current item. There are 5 other text boxes which are for the length, width, height, weight, and number of items.

\figur{0.2}{AddSuitcase.png}{Add suitcase window}{fig:AddSuitcase}
When the user presses the "Add Suitcase" button in Manage Suitcases, the user can add a new suitcase. The data needed is the length, width, height, weight, and the maximum weight of the suitcase. 

\figur{0.2}{EditItem.png}{Edit item window}{fig:EditItem}
If the user wants to edit an item, he/she can press the "Edit Item" button in Manage Items, and this form shows. In the form there are 6 text boxes for each input parameter, and a button saying "Edit", which saves the changes the user has made and closes the form.

\figur{0.2}{EditSuitcase.png}{Edit suitcase window}{fig:EditSuitcase}
In "Manage Suitcase", there is a button called "Edit Suitcase". It allows the user to change the data of a suitcase, if e.g. the measurements are wrong, or the user wants to use another suitcase, which do not have the same measurements.

\figur{0.5}{LoadSavedLists.png}{Load List}{fig:LoadList}
If the user already have used the program before and have saved an item list and a suitcase list, both can be loaded here.

\subsection{3D-viewer}
The 3D-viewer shows how the program have packed the different items in the suitcases.

\figur{0.5}{frm3DViewer.png}{3D Viewer window}{fig:3dViewer}

This is the first thing the user will see when the packing starts. It shows how the items are placed in the suitcase.
The image can be dragged, moved, and zoomed with the mouse, as seen below. When the user clicks on an item in the list on the right side, the marked item will be highlighted in the image. Below the list are the xyz-points to see where the item is supposed to be placed, and the current suitcase's weight. 

\figur{0.5}{frm3DViewer2.png}{3D Viewer2 window}{fig:3dViewer2}

On the left side of the window are two buttons, zoom in and zoom out. On the lower left side of the window is a check box called "Zoom limit". It sets the limit for how close and how far the user can zoom the image. The buttons have been made in the case the user does not have a mouse with a scrolling wheel or is on laptop. The track bar on the left is a tool to adjust the speed when the user rotates the image. The reset button resets the track bar. Below the reset button is a drop-down list containing the suitcases the user has packed items into.

The reason why it looks like it does is because it gives a good overview with the item list on the right, the 3D-image in the middle, and the image options (zoom in, zoom out, reset etc.) on the left.

In the 3D-image a small box has been made in point(0;0;0) to indicate where it is, so the user easier can navigate through the items.