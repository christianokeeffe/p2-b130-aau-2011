\chapter{Pak kufferten}
\label{chap:forslag}
\section*{Problemstilling}
Hvis man skal ud at rejse med overnatning, skal man huske at pakke sin kuffert. Det kræver en del overvejelser. Det går ikke at glemme sit pas eller sit undertøj.
Men et kuffertpakningsprogram skal kunne klare meget mere end at være en avanceret huskeseddel. Et godt program skal også kunne hjælpe med at fordele bagagen i fordel til vægt- og volumenkrav.
På mange flyselskaber er der således krav til den indtjekkede bagage og håndbagagen. Mange flyselskaber tillader 20 kg indtjekket bagage og har krav til dimensionerne på håndbagage. Bestemte slags genstande må ikke forefindes i håndbagagen, andre ikke engang i indtjekket bagage.
Hvis man er en familie, der skal rejse sammen, er det vigtigt at fordele indholdet i kufferter, så de mindste medlemmer ikke slæber for meget, men heller ikke for lidt. Også volumen er en udfordring, hvis man skal rejse i bil.
Hvilke kufferter og kasser er der plads til i bagagerummet? Og hvordan skal bagagen placeres i bagagerummet? Man kan selvfølgelig prøve sig frem, men det ville være godt at have en algoritme til hjælp. Ideelt set skulle et pakkeprogram være en lille app, som man kan have med sig i sin telefon eller tavlecomputer og let tilgå undervejs på rejsen, når man f.eks. skal finde plads til den nyindkøbte souvenir eller de fem flasker etellerandet.
\section*{Problem}
Kan man lave et program, der kan hjælpe med at planlægge pakning af kuffert?
\section*{Formål}
At udvikle et stykke software, der kan være en intelligent assistent, når man skal pakke sin bagage.
\section*{Mål}
At udvikle et effektivt stykke software, der gør det muligt at pakke sin bagage
bedst muligt.
\section*{Teknisk-naturvidenskabelige fagområder}
Optimeringsproblemer. Analyse af algoritmer. Objektorienteret programmering.
\section*{Eksempler på kontekstuelle fagområder}
Hvorfor er luftfartsselskabernes pladskrav opstået, og hvordan har de udviklet
sig?
\section*{Forslagsstiller}
Hans Hüttel.