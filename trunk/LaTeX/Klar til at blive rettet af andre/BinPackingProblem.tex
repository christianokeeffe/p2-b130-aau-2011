\documentclass[a4paper,11pt,fleqn,twoside,openright]{memoir} % Brug openright hvis chapters skal starte på højresider; openany, oneside

%%%% PACKAGES %%%%

%  Oversættelse og tegnsætning  %
\usepackage[utf8]{inputenc}					% Gør det muligt at bruge æ, ø og å i sine .tex-filer
%\usepackage[danish]{babel}							% Dansk sporg, f.eks. tabel, figur og kapitel
\usepackage[english]{babel}
\usepackage[T1]{fontenc}  % Hjælper med orddeling ved æ, ø og å. Sætter fontene til at være ps-fonte, i stedet for bmp					
\usepackage{latexsym}										% LaTeX symboler
\usepackage{xcolor,ragged2e,fix-cm}			% Justering af elementer
\usepackage{pdfpages} % Gør det muligt at inkludere pdf-dokumenter med kommandoen \includepdf[pages={x-y}]{fil.pdf}	
\usepackage{fixltx2e}					% Retter forskellige bugs i LaTeX-kernen

																			
%  Figurer og tabeller floats %
\pdfoptionpdfminorversion=6	% Muliggør inkludering af pdf dokumenter, af version 1.6 og højere
\usepackage{graphicx} 		% Pakke til jpeg/png billeder
	
%  Matematiske formler og maskinkode 
\usepackage{amsmath,amssymb,stmaryrd} 	% Bedre matematik og ekstra fonte
\usepackage{textcomp}                 	% Adgang til tekstsymboler
\usepackage{mathtools}			% Udvidelse af amsmath-pakken.
\usepackage{siunitx}			% Flot og konsistent præsentation af tal og enheder med \SI{tal}{enhed}

%  Referencer, bibtex og url'er  %
\usepackage{url}	% Til at sætte urler op med. Virker sammen med hyperref
%\usepackage[danish]{varioref} % Giver flere bedre mulighed for at lave krydshenvisninger
\usepackage[english]{varioref} % Giver flere bedre mulighed for at lave krydshenvisninger
\usepackage{natbib}	% Litteraturliste med forfatter-år og nummerede referencer
\usepackage{xr}		% Referencer til eksternt dokument med \externaldocument{<NAVN>}
\usepackage{nomencl}	% Pakke til at danne nomenklaturliste
\makenomenclature		% Nomenklaturliste

%  Floats  %
\let\newfloat\relax 	% Memoir har allerede defineret denne, men det gør float pakken også
\usepackage{float}
%\usepackage[footnote,draft,danish,silent,nomargin]{fixme}	% Indsæt rettelser og lignende med \fixme{...} Med final i stedet for draft, udløses en error for hver fixme, der ikke er slettet, når rapporten bygges.
\usepackage[footnote,draft,english,silent,nomargin]{fixme}

%%%% CUSTOM SETTINGS %%%%

%  Marginer  %
\setlrmarginsandblock{3.5cm}{2.5cm}{*}	% \setlrmarginsandblock{Indbinding}{Kant}{Ratio}
\setulmarginsandblock{2.5cm}{3.0cm}{*}	% \setulmarginsandblock{Top}{Bund}{Ratio}
\checkandfixthelayout 

%  Litteraturlisten  %
\bibpunct[,]{[}{]}{;}{a}{,}{,} 	% Definerer de 6 parametre ved Harvard henvisning (bl.a. parantestype og seperatortegn)
\bibliographystyle{bibtex/harvard}	% Udseende af litteraturlisten. Ligner dk-apali - mvh Klein

%  Indholdsfortegnelse  %
\setsecnumdepth{subsection}	% Dybden af nummerede overkrifter (part/chapter/section/subsection)
\maxsecnumdepth{subsection}	% Ændring af dokumentklassens grænse for nummereringsdybde
\settocdepth{subsection} 		% Dybden af indholdsfortegnelsen


%  Visuelle referencer  %
\usepackage[colorlinks]{hyperref} % Giver mulighed for at ens referencer bliver til klikbare hyperlinks. .. [colorlinks]{..}
%\usepackage{memhfixc}
\hypersetup{pdfborder = 0 0 0}	% Fjerner ramme omkring links i fx indholsfotegnelsen og ved kildehenvisninger 
\hypersetup{			%	Opsætning af farvede hyperlinks
    colorlinks = false,
    linkcolor = black,
    anchorcolor = black,
    citecolor = black
}

\definecolor{gray}{gray}{0.80}					% Definerer farven grå

%  Kapiteludssende  %
\definecolor{numbercolor}{gray}{0.7}			% Definerer en farve til brug til kapiteludseende
\newif\ifchapternonum

\makechapterstyle{jenor}{			% Definerer kapiteludseende -->
  \renewcommand\printchaptername{}
  \renewcommand\printchapternum{}
  \renewcommand\printchapternonum{\chapternonumtrue}
  \renewcommand\chaptitlefont{\fontfamily{pbk}\fontseries{db}\fontshape{n}\fontsize{25}{35}\selectfont\raggedleft}
  \renewcommand\chapnumfont{\fontfamily{pbk}\fontseries{m}\fontshape{n}\fontsize{1in}{0in}\selectfont\color{numbercolor}}
  \renewcommand\printchaptertitle[1]{%
    \noindent
    \ifchapternonum
    \begin{tabularx}{\textwidth}{X}
    {\let\\\newline\chaptitlefont ##1\par} 
    \end{tabularx}
    \par\vskip-2.5mm\hrule
    \else
    \begin{tabularx}{\textwidth}{Xl}
    {\parbox[b]{\linewidth}{\chaptitlefont ##1}} & \raisebox{-15pt}{\chapnumfont \thechapter}
    \end{tabularx}
    \par\vskip2mm\hrule
    \fi
  }
}			% <--

\chapterstyle{jenor}	% Valg af kapiteludseende - dette kan udskiftes efter ønske


%\renewcommand{\headrulewidth}{0.4pt}
%\renewcommand{\footrulewidth}{0.4pt}


% Sidehoved %

\makepagestyle{custom} % Definerer sidehoved og sidefod - kan modificeres efter ønske -->
\makepsmarks{custom}{																						
\def\chaptermark##1{\markboth{\itshape\thechapter. ##1}{}} % Henter kapitlet den pågældende side hører under med kommandoen \leftmark. \itshape gør teksten kursiv
\def\sectionmark##1{\markright{\thesection. ##1}{}}	% Henter afsnittet den pågældende side hører under med kommandoen \rightmark
} % Sidetallet skrives med kommandoen \thepage	
\makeevenhead{custom}{Gruppe B205}{}{} % Definerer lige siders sidehoved efter modellen \makeevenhead{Navn}{Venstre}{Center}{Højre}
\makeoddhead{custom}{}{}{Aalborg Universitet} % Definerer ulige siders sidehoved efter modellen \makeoddhead{Navn}{Venstre}{Center}{Højre}
\makeevenfoot{custom}{Page \thepage}{}{}													% Definerer lige siders sidefod efter modellen \makeevenfoot{Navn}{Venstre}{Center}{Højre}
\makeoddfoot{custom}{}{}{Page \thepage}														% Definerer ulige siders sidefod efter modellen \makeoddfoot{Navn}{Venstre}{Center}{Højre}		
\makeheadrule{custom}{\textwidth}{0.5pt}	 % Tilføjer en streg under sidehovedets indhold
\makefootrule{custom}{\textwidth}{0.5pt}{1mm}	% Tilføjer en streg under sidefodens indhold

\copypagestyle{nychapter}{custom}														% Følgende linier sørger for, at sidefoden bibeholdes på kapitlets fåøste side
\makeoddhead{nychapter}{}{}{}
\makeevenhead{nychapter}{}{}{}
\makeheadrule{nychapter}{\textwidth}{0pt}
\aliaspagestyle{chapter}{nychapter}													% <--

\pagestyle{custom} % Valg af sidehoved og sidefod


%%%% CUSTOM COMMANDS %%%%

%  Promille-hack (\promille)  %
\newcommand{\promille}{%
  \relax\ifmmode\promillezeichen
        \else\leavevmode\(\mathsurround=0pt\promillezeichen\)\fi}
\newcommand{\promillezeichen}{%
  \kern-.05em%
  \raise.5ex\hbox{\the\scriptfont0 0}%
  \kern-.15em/\kern-.15em%
  \lower.25ex\hbox{\the\scriptfont0 00}}

% Billede hack %
\newcommand{\figur}[4]{
		\begin{figure}[H] \centering
			\includegraphics[width=#1\textwidth]{billeder/#2}
			\caption{#3}\label{#4}
		\end{figure} 
}

%%%% ORDDELING %%%%

\hyphenation{hvad hvem hvor}
\begin{document}
\section{Bin packing problem}
\label{sec:binpacking}
Bin packing problems is a combinatorial NP-hard problem. The problem consists of fitting objects of different sizes into bins of identical sizes \ref{appofdismath}. This could for example be fitting various packages into shipping containers. There are various approaches to solve the bin packing problem. Bin packing problem is focusing on bins instead of suitcases but they are basically the same only major different is probably size. Some of the popular methods will be described in the following section. To describe these packing algorithms, illustrations will be used. The illustrations show how the packing algorithms work in one dimension - but it gives a nice overview of how the algorithms works. Figure  \ref{bpini} is an illustration of the unpacked elements:
\figur{0.7}{initial.png}{Initial elements}{bpini}



\subsection{First fit (FF)}
The first fit algorithm creates a list of the objects needed to be fitted into bins. It then runs through the list, checking if an item can fit in each bin: If it cannot fit in the first bin, it will check if it can fit in the second bin and so on. If it does not fit in any bins, it opens a new bin, and fits the object there. Figure \ref{bpff} is an illustration of the elements packed with the First fit algorithm.
\figur{0.7}{ff.png}{Elements after FF has been applied}{bpff}


\subsection{Best fit (BF)}
The best fit algorithm is the same as the first fit algorithm, except that before an object is packed, the algorithm checks each open bin, where the object fit. It will then place the object in the bin which will have the least space left when the object is packed. Figure \ref{bpbf} is an illustration of the elements packed with the Best fit algorithm.
\figur{0.7}{bf.png}{Elements after BF has been applied}{bpbf}

\subsection{Last fit (LF)}
This algorithm packs the object in the last open bin which has room for it. This algorithm is thereby the opposite of the first fit algorithm. Figure \ref{bplf} is an illustration of the elements packed with the Last fit algorithm.
\figur{0.7}{lf.png}{Elements after LF has been applied}{bplf}


\subsection{Worst fit (WF)}
The algorithm checks all the bins, and packs the object in the bin which has most empty space. As its name suggest, this algorithm is the opposite of the Best fit algorithm. Figure \ref{bpwf} is an illustration of the elements packed with the Worst fit algorithm. As the figure shows, the worst fit algorithm is in fact more effective than its name might suggest. 
\figur{0.7}{wf.png}{Elements after WF has been applied}{bpwf}

\subsection{Almost worst fit (AWF)}
Similar to the worst fit algorithm, but the almost worst fit algorithm packs the object in the second-emptiest bin.  Figure \ref{bpawf} is an illustration of the elements packed with the Almost worst fit algorithm.
\figur{0.7}{awf.png}{Elements after AWF has been applied}{bpawf}

\subsection{First fit decreasing(FFD)}
The above algorithms are very ineffective because the biggest objects might be placed at the end of the list, and thus be packed in the end, where it is more effective to first pack these large objects.
The first fit decreasing algorithms takes this into account and sorts the list before attempting to pack the items. This way the biggest items will be packed first.

\subsection{Best fit decreasing(BFD)}
Again this is the same as the best fit algorithm, but with the list being sorted before attempting to pack the objects.

\subsection{Round up}
It seems that it is more effective to sort the lists before attempting to pack objects into bins. This way bigger objects are packed first, and the smaller objects can then be fitted around the bigger objects. However in some situations it is necessary to use unsorted lists. For example in a factory with continuous production, it is never possible to have the complete list of objects, and thus never possible to sort the list.

\end{document}