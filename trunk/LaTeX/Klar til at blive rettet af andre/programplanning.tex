\section{Program planning}
This section is to plan how the program should work and describe the flow of the program. The program will be described in a flowchart to give an overview of the whole program. A flowchart is a useful tool when programming because it explains the structure of the program.

To give a more precise explanation of a program the flowchart can be formed into a pseudo code which is a level above real code. Pseudo code is used as a schematic for the program and thereby give some foresight into any problems that can be encountered when writing the actual code. Thus planning ahead and designing the program so a minimum amount of code errors and unexpected problems occur.
The program planning will be used to make it easier to develop the program and help make a better product in terms of structure.

When the program starts, it should show the main window. Here can the user load saved lists, manage the item list or the suitcase list, the instructions are shown, there is a button that will show information about the program and the button that will start the packing of the suitcase. In the manage windows the user are able to clear the list, add new items, edit items, delete items, and save the list. If one or both of the lists are empty when the user clicks the "Start packing"-button the program will inform the user that there needs to be atleast one item and one suitcase for the program to be able to pack the suitcase(s) and item(s). After managing the list the user are asked to click the "Start packing"-button.

The program then preforms the algorithms to place the items in the most efficient way regarding volume and weight. The program will also check that the suitcases does not exceed the weight limit set by the user.
When the program successfully place an item, the item will be marked as packed. If the program can not fit the item in any of the accessible suitcases the item will be marked as not packable. If the program reach the point where all items have gone through the process, it should then inform the user that the process is done and inform how the user have to pack the suitcases and report if there were any items that could not be packed.
At the end of the program the user will be able to see a 3D-viewer of the suitcase and be able to select which suitcase to show. The user will be able to zoom, rotate and drag the suitcase. A list will show all the items in the suitcase and by selected they will be marked in the suitcase.

\figur{0.9}{flowchart.JPG}{This is the flowchart of the program (not quiet finished yet)}{fig:flow}

Hereby the general structure of the program has been formed and can be described by a flowchart, seen on figure: \ref{fig:flow}.
The arrows shows the direction of the flow in the program. Some of the arrows also have small labels indicating what answer there were to the decision.
This flowchart can then be used as a schematic for the developing of the program and thereby a better structure of the program can be achieved.