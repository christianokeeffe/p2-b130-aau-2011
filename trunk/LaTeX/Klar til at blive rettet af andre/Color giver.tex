\section{Color giver}
This function gives each item a color and make sure that items that are next to each other do not get the same color. The function have four steps to complete before its task is finish.\\
\\
The first step is to add all the item's points into a list, this is done for all items that are to be packed. First off a "for" loop runs through the items that are to be packed. Inside the first "for" loop there are three "for" loops that runs through the x, y and z values starting at the saved point for the current item. The loops then runs until the limit is reached for the three "for" loops which is the height, width and length. In the last of the three "for" loop there is a "if" statement that checks that it is only coordinates(x, y, and z) on the outside of an item that are added. When the "if" statement is fulfilled the three integers x, y and z are added to their representative lists that are stored in a temporary variable of type cbox. These three lists outlines the sides of their respective object. The temporary variable are then stored in the BoxColor list which is the finally list.\\ 
\kode{the 4 "for" loops that controls the item and the x, y, z value that make the point for the items}{colorGiverListing1}{colorGiverListing1.txt}
\\
The second step is to check which items that are neighbor through two "for" loops. The 2 "for" loops checks whether or not the two given items are in the same container and if they are close to each other. This is done by comparing the two items and see if the two items shares any points given by the three coordinates(x, y, z). It works by checking if one of the coordinates (x, y or z) is either at the others start point or at the maximum length and the rest of it check if the rest of the coordinates are around the item. Then the temporary item is added to the main list of items.  The temporary item is reset just before a new item is made so the item do not get the previous points into them.\\
\\
Next step is to check if points from two items are the same. Thereby determine if they are located next to each other, if they are, the item number will be saved in a list that is called neighbour. This step works by 4 "for" loops where the first one keeps track of the current item that is being checked for neighbours, the second keeps track of the item that is being checked if it is a neighbour to the first one. The last 2 work with the points that are in the items that are being worked with. But it will only run these loop if the items are in the same suitcase(Which can be seen by the "if" sentence that are between the 2 first and the 2 last "for" loops). If they are next to each other it will save the item number into a list in the other item this list is done for both items that are checked.\\
\\
Thereafter the list "neighbour" are gone through to remove all the duplicates of themselves from the list. This work with one "for" loop that keeps track of which item that are being worked with and when the loop is done the neighbour list is sorted. Then there is a "while" loop that go through all the neighbours(Items that are next to it) and check if the item itself is in it and remove it. It will also check if the number after the current is the same, if it is it will remove one of them.\\
\\
The last step taken is to go through the item list and give a color to each item. In this part the colors are numbers, where number 0 is the standard color. First is a "for" loop that that sets the current item. Before the current item is given a number based on its neighbours, its value is set to 0. Then there is a "while" loop that go through all the item's neighbours and check if the color is the same. If they are the same then the item's color number is increased by 1. When the neighbour list is finished the color is saved in the item, and the next item is checked.