\section{Discussion}
The focus in this discussion is to see if the product is in conjunction with the problem analysis and to see if the different requirements is integrated in the product.

In the problem analysis there was stated the following requirements for the program:

\begin{itemize}
\item Program language is C\#.
\item Guide the user.
\item Distribute weight.
\item Distribute space.
\item On the road.
\item Baggage rules.
\item Structure of packing.
\item Packing list.
\item save/load function.
\end{itemize}

And there also were some Optional Features that if there where time left those might be implemented in the program:
\begin{itemize}
\item Solid/liquid/bendable shapes.
\item Type of trip.
\end{itemize}

The program is based on the requirements and has been improved using the tests that have been made on the program. The tests have helped with improving the design of the program and a few function in the program. This has been a very good tool in improving and bug-testing the program. The improvement made after the tests can be seen in chapter \ref{chap:testing}.

In section \ref{sec:devalgorithm} the algorithm are explained. The algorithm distribute the items in the suitcase(s) evenly by weight and space. The algorithm also make sure that the weight limit is not exceeded. Yet the algorithm can not pack as good as a person might be able to do. One reason that a person might be better at packing the suitcase,, is that the program take all the items as boxes and do not take into account, that some items might be able to be fold and bend in different shapes.

To help the user to get a better visual understanding on how to pack the suitcase a 3D-image is made for the user that can rotate, zoom and be dragged. A list has been added in the right side showing a list of all the packed items in the selected suitcase. By selecting an item it will be marked in the 3D-image so the user will easily be able to identify any given item. An explanation of this function can be seen in section \ref{3DHandle}. To help the user furthermore an algorithm has been made that makes sure that any two items next to each other will have different colors, so they are easier to identify from each other. The function and a description of it can be seen in section \ref{sec:ColorGiver}.

It is easy to get a good overview in the program, because the maximum number of buttons is 5 in all the different windows and no unnecessary features are in the windows. The 3D-viewer is very user friendly because the user can use the mouse to zoom and rotate the 3D-image of the suitcase containing the packed items. The section where the GUI is described is in section \ref{sec:GUI}.

The user are able to save a list he/she has made on the computer and load saved list every time he/she uses the program.

\fxfatal{det der står under skal skrived om på.}
-------------------------------------------------------------------------- 

The user can make or save into a file that also can be loaded by the program to get the data that are in the file. There is different way to save data one is to use windows SQL database that where not ideally because it need a SQL handling program or internet connection to use. Therefore there is use a File serializer that make a file with the data. This is useful because it make it easily to choose where the data should be save into the data file or with data file should be load into the program.