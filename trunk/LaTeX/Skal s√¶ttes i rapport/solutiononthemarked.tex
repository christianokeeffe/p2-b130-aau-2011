\section{Solutions on the market}
\label{sec:Solution}
This chapter are used to research the marked and thereby get a image of what solutions there already exits on the marked. Through the research it is also possible to determine, how the exiting solution helps the user with the stated problem. By looking at existing solutions it can used to determine what features that would needed in an more problem oriented solution.
The amount of lists and guides on the market is huge. These lists and guides offers help and provide tips for packing for travelling. Some of these lists and guides have been developed into applications that are available for the customer to use.
An application or in short app, is a program that fulfil different kind of services for the user. Apps are used in the web browser, computer, smart phone, and tablets. The term app is general mentioned in context to smart phones and tablets.

There also exists programs, that have integrated algorithms to handle optimization of the packing, on the marked that can be used.
First a look into these lists and guides and the more advance solution thereafter.

\subsection{App - Packing Pro}

Packing pro is an app developed for the Iphone and Ipad that offers templates for check lists to the customer. Iphone is a smart phone and the brand is owned by Apple. Packing Pro uses a touch interface which means that the user by using the finger can navigate around. This is possible because Iphone and Ipad support touch navigation.

\figur{0.8}{Iphone.JPG}{Picture of 2 of Packing Pros menus from \citep{ipack}}{fig:ipack}

Packing pro is designed with a panel in the bottom of the screen that allows the user navigate through the menus. Packing Pro provides the user with a help menu that contain information on how to use the app. There is also a menu in where the user can charge color theme of the app. On figure: \ref{fig:ipack} can an example of how a check list could look like.

These templates are designed to different purposes regarding the customer, gender, type of trip, and purpose of the trip.
The customer can then load the wanted template for the purpose. The user also have the possibility to create their own lists by adding things that should be remembered for the trip by them self. The user also have possibility to select an existing list and delete the objects that were found irrelevant by the user. The user can then check object on the lists off as it get packed. Packing Pro is a management tool that helps the customer get an overview of all the things to remember. As the name implies(pro) the app have to be bought before it can be used  \citep{packingpro}.

Packing Pro works as check list and help the user remember what to pack, but it does not preform any organization of luggages content itself. So Packing Pro itself does not solve the described problem but helps the user remembering what to pack. So a feature to use from this program is the check list function that gives the user an overview of things to pack.
A function that would not be needed is the compatible with Iphone/Ipad operating system. It would be nice if the program made work cross platforms but it is still not required to solve the problem.

\subsection{App - Checkmark Packlist}

Checkmark Packlist is an free app for the smart phones running the Android system.
Checkmark Packlist offers different templates for check lists that the customer can use. One of these templates is the list for packing for a trip. That way the customer can select and use this template for remembering what they will need to pack for the trip. Checkmark Packlist uses touch to navigated in the program. This means that the customer with their fingers can navigated through the check list and check off things that have been packed.

\figur{0.4}{checkmark.JPG}{Picture of the Checkmark Packlist in action from \citep{checkparklist}}{fig:checkmark}

On figure: \ref{fig:checkmark} there can be seen an example of the product and how Pheckmark Packlist looks like for the user.
Checkmark Packlist does not provide customization tool that let the customer add more categories to the check list. This is only a featured provided in the paid version of Checkmark Packlist \citep{checkpacklist}.

This app does not provide a solid solution to the found problem but helps the user remembering what there should be packed for trip. This check list feature give a guiding effect and this is a useful feature and can be used in the product design to solve the state problem. A feature to consider is the mobility by designing the program for hand held devices.

\subsection{Online check/tip list}

The online check list works as a reminder when packing luggage. It also give tips and tricks that could be considered when packing for the trip. There exists a lot of different websites offering this service for free. Some are posted by an organization and others by a person on a forum. This means that all electronic devices as computers, tablets, and smart phones that have access to the Internet can open the website address.

An example of this kind of website is the following source \citep{onlinecheck}. This website offers a list of 10 tips that can be helpful for the customer when they are packing for a trip. The website is purely text based and helps the user packing through the tips on the website.
The site does not help with the actual packing, instead it helps with the planing of materials that the user might want to have on the trip.
The website is designed with a menu left that let the user navigate through the different content of the website.

The online check/tip list in itself does not give the customer a solution to the packing problem. The websites instead help the customer planing the trip and thereby no the actual problem. The type of check/tip list used on \citep{onlinecheck} does not apply as a useful feature that could be used in the final program. Instead it would more be focus on helping the user with the packing.

\subsection{The e-Commerce shipping calculator}

The e-Commerce shipping calculator is an advanced program that helps the customer packing large containers and calculates the price of the shipment.
By typing the size, weight, location, and destination of the items that should be shipped, the program can calculate what the prize is going to be and generates a 3D(3 dimensional) model of the container where the given items are placed in the best possible way so there are a minimum of wasted space. On their website \citep{solvingmaze} they offer a demo(demonstration) of their program. Their demo runs through the web browser and thereby should be accessible from computers connected to the internet.
The demo is design to have the containers dimension variables and weight limit as input fields. Under the container is there are list of item where each item can have different dimensions and weights. To right of these field is the 3d model placed that will be generated.

\figur{0.6}{shippingcalculator.JPG}{Screen shot of the program running taken from \citep{solvingmaze}}{fig:calculator}

The customer have to type all known data in and press "Calculate Rates" and the program will then form a 3d model, this can be seen on figure: \ref{fig:calculator}.
This product have a number of useful features that can be used in the final product. This solution can take items dimensions and weight and calculate the most optimal placement in the container. This can be related to packing your luggage for a trip. So a feature to have in the final product is to  calculate somewhat most effective way to pack the users luggage. Another useful feature is that this solution also shows it to the user. So the final product most inform the user in similar or another way how the luggage should be pack.

\subsection{Recapitulation}

This sections main object is to look at the wanted features and recap them.
Packing Pro and Checkmark Packlist is similar in the way that they help the user. They both provide the user with a sort of check list that can be check off and thereby help the customer remember what have not been packed yet.
One of the differences is that Packing Pro have a price while Checkmark Packlist is free. But this difference means that Packing Pro allows the user to edit the check list while that are not possible in Checkmark Packlist.
These two apps do not help the user arrange the luggage content or take in consideration of size and weight of it. Thereby is two apps do not help people with the packing itself but more what should be remembered for the trip.
The online check/tip list provide the user with advices for the trip and what to pack. Advices are great to get a general idea of what to take pack but it still does not give a more effective way to pack.
The e-Commerce shipping calculator is the one solution with the most wanted features. One of the strong feature that can be used is the ability to calculate the most effective that a container should be packed. A important side note is that the intentions is not to pack bags but the feature can be related to packing content of a bag.

\begin{table}[H]
\begin{center}
\begin{tabular}{c  c | c | c | c | c | c}
\textbf{Included in product} &  \rotatebox{90}{\textbf{Solutions}} &\rotatebox{90}{App - Packing / Packing Pro} & \rotatebox{90}{App - Checkmark Packlist}& \rotatebox{90}{Online check/tip list}&\rotatebox{90}{The e-Commerce shipping calculator}\\ \hline
Guide the user & & x & x & x & x   \\ \hline
Distribute weight &  &   &   &   & x    \\ \hline
Distribute space  &  &   &   &   & x    \\ \hline
On the road   &  &  x  &  x  &   &  x  \\ \hline
Baggage rules  &  &    &    &   &   \\ \hline
Where in the suitcase  &  &   &   &   &  x   \\ \hline
Packing list &  & x & x &   & x    \\ \hline

\end{tabular}
\end{center}
\caption{ Table for the different products on the market compared to features}
\label{tab:OtherPrograms}
\end{table}

Table \ref{tab:OtherPrograms} consists of features down the y-akse and exiting solution on the market out the x-akse. The crosses indicates when the particular product have the particular feature. The purpose of the table is to give a overview of the products compared to features that was found essential to the problem.
Table \ref{tab:OtherPrograms} shows that none of the exiting solution does not take rules regarding baggage. It also shows that there are a lot of help regarding what to bring but when it comes to packing it is only one of the selected solution that had this feature.