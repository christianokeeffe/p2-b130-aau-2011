\section{Target Group Analysis}
The target group analysis for this project will be based on the Minerva Model, because it is an internationally recognized model, and it suits the international market better than e.g. the Gallup Compass.

The Minerva model consist of 4 main segments, and 1 sub segment; The blue, the green, the pink, the violet, and the grey.
\begin{itemize}
\item The modern/materialistic segment(blue)

These people are individualists who mostly believe in themselves and are rarely solidary with certain groups, but they gladly join other people who are going the same way as they are. The crucial key for the modern materialist's group is that they consider society as a relatively fair and justified system which rewards the one who does an effort that can be felt in society.

The blue segment mostly consists of men with a long education and a high wage. The Danish Liberal Party stay true to their original values from when the party was established, this makes them the favourite party among the blue segment. It is estimated that 25\% of the Danish population belongs in the blue segment.

\item The modern/idealistic segment(green)

The green segment consists of modern, but idealistic, and strongly solidary people who have the same kind of education as themselves. They often consider society as something that is "higher" than the group they belong to, and they consider everything outside their own group with some scepticism. Depending on the conjunctures they seem to have some tendency to feel either better than society or besieged.

This segment consists of mostly women, academics, and similar. They often read "Information" or "Politiken", two left oriented newspapers. They Vote for The Danish Socialistic Folk Party (Socialistisk Folkeparti), The Danish Red-Green Party (Enhedslisten), The Danish Radical Left Party (Det Radikale Venstre), and they are courted by The Danish Social Democrats (Socialdemokraterne).
It is estimated that 25\% of the Danish population belongs in the green segment.

\item The traditional/idealistic segment(pink)

This culture is traditional-idealistic, and is in many ways, the most "original". In this segment they are for-anchored in local areas and feel a strong solidarity with everyone in the neighbourhood. Within this culture, one's view upon charity for close friends and family should be taken literal, and the world around them become more and more frightening the further you move away from it.

The people in this group tend to be uneducated, or at least have a short education. Some of them have an allotment garden, and other Danish valuable stuff which they took great care of. They have earlier voted for The Danish Social Democrats, but now it is The Danish Folk Party (Dansk Folkeparti) that is the most favourable party.

It is estimated that 20\% of the population belongs in the pink segment.

\item The traditional/materialistic segment(violet)

People in this segment have a lot of the same norms as the pink segment, but have lost their roots, which is why it often seems to miss some orienting points in life, both in the relation to the group and in relation to society. As a compensation the violet person seeks exile in consumption, and modern offers about group affiliation such as associations, hobbies, sports clubs, fan clubs etc. 

The violet have, as the pink, lost group affiliation to the working party and are now those who can not decide whether to vote The Conservative or The Danish Folk Party, but they definitely have their favourite football club. They frequently seek physical challenges rather than intellectual, and work spare time jobs.

It is estimated that 20\% of the population belongs to the violet segment.

\item The undecided segment(grey)

The grey does not appear in A.C. Nielsen's official model and are the most discussed individuals. It is the people who have not decided which segment they belong to yet. It is typically the youth, but some adults have not found themselves in a specific segment. The grey segment is 10-11\% of the population above the age of 18.

The grey can be found in the middle of coordinate system and contains features from all the "pure" segments.
\end{itemize}
\citep{minerva}
\figur{1.0}{Minerva-modellen.jpg}{The Minerva Model}{fig:minerva}

As the Minerva model is based on political views and therefore not the most effective method to define the target group for this project. The reason to this conclusion is that packing a suitcase is not a political matter.
Still the Minerva can be used to create a general base for the target group using the general social characteristics of the different segments.
The targeted group cannot be put into just one segment because there are people from both the blue, the green, and the grey segments. It is a quite wide group, but it might be necessary for the product to cover more than one segment.
The target group for this project is in the modern segments because it has been estimated that the users are somewhat interested in new technologies, and so there is a chance of finding somebody in this segment who do not know how to properly pack a suitcase.
A requirement for target group is that they must travel in some form, that requires a packed suitcase or luggage in some form, before the solution becomes relevant for the target group.