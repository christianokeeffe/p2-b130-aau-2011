\section{Specification Requirements}
\label{sec:Spec}
The program has some features, which are essential for the program to work, these are catalogued as "targeted features". The other features, the program does not necessarily need to work, is catalogued as "Nice to have" features.
\newline

\subsection{Targeted Features}
These are the features that the program will have. 


\textbf{Written in C\#}:
The requirements of the project is, that the solution must be a program written in C\#. It must have a grapical user interface (GUI).

\textbf{Guide the user}:
The program will have a little readme file, or other form of guide, that will tell the customer how to use the program..
\newline

\textbf{Distribute weight}:
The program must 
\newline

\textbf{Scaling}:
When the pinpoint has been made the program will then crop the picture so the head, shoulders, and part of the background is all that is left. Then it will scale the picture to the size of a passport photo, if needed.
\newline

\textbf{Display image}:
As the user opens the program and has successfully renamed the file, the image will then pop up. This will make it easier for the users to know what is going on, and what they have to look for in the picture.
\newline

\textbf{Dots per inch}:
For the photo to be accepted as an approved passport photo, it has to meet certain quality specifications. So the program will also have a feature that determines whether or not the photo is of a good enough quality. By measuring the dot per inch (DPI) the program will tell the customer if it is not high enough quality, and then the photo will not be approved.
\newline

\textbf{Checking}:
When the photo is done the program will show a preview to the user so he/she can go through a number of questions to help check the photo to determine whether the rules for passport photo are met.
\newline

\textbf{Finish}:
This is the last part of the program. It is simply a little preview of how the picture looks, when it is done without all the guidelines.
\newline


\subsection{Optional Features}
The program does not take into account that an image might be turned to one side. So a "nice to have" feature would be a rotate function that will solve this problem. The program of this project is based on European requirements for photos, but some countries have different requirements, so to further increase usability a plug-in or a built-in feature to choose from different countries would be preferable. Furthermore, due to the bit-structure of black and white images they have not been included in this project.

\subsection{Will not Have}
Given the limited visual functions of C programming, the program will not have a graphical user interface, only lines which to write on. Furthermore, the program is intended as substitute for an image expert, so it will help the user check whether or not the photo follows the requirements. For the same reason the program will not print the photo.
