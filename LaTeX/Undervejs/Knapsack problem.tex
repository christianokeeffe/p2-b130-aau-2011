\section{Knapsack Problems}
\label{sec:knapsack}
The knapsack problem is basically creating an algorithm which packs a list of items into a knapsack. Each item is assigned a weight and a value. The total weight of the items must not exceed the maximum weight capacity. At the same time, the knapsack must be packed so the summarized values of the items are as high as possible. There exists a vast amount of derivatives of the knapsack problem. For example the 0-1 knapsack problem\ref{knapsackproblems}, which dictates that each item can only have the status 1 or 0, which equals packed or unpacked. This means that each item can only be packed once, where in the regular knapsack problem, items can be packed multiple times to maximize the total value of the knapsack. A knapsack problem can can be formulated as the solution to the following linear integer programming formulation:


\[maximize \sum\limits_{j=1}^n p_jx_j\]
Meaning: Maximizes the total value (p) of items (j) in knapsack

\[subject~to \sum\limits_{j=1}^n w_jx_j\leq c,\]
\[x_j \epsilon \left \{ 0,1 \right \}, j=1,...,n.\]
Meaning: Total weight (w) of items(x) may not exceed the knapsacks capacity (c)

 \ref{knapsackproblems}
\newline
Another derivative is the bin packing problem, which will be described in the following section.