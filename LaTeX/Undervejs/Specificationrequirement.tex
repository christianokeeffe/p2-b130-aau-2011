\section{Specification Requirements}
\label{sec:Spec}
Through the problem analysis (see chapter \ref{chap:problem} it has been documented that there are some strict rules regarding some forms of public transportation when going on vacation \fxfatal{missing section reference}. Based on this research a list of features has been composed, which the program must fulfil to meet the base requirements \fxfatal{missing section reference}. Furthermore another list has also been made composed of some additional features that would make the program better and more user friendly. They are not needed for the base requirements, but rather as improvements to make the program even more ideal for the user.

Since using a new program sometimes can be hard, it is necessarily to have a guide or some instructions to the user to help him/her using the program the way it was designed. The guide should be short and well formulated so the user with ease can understand what to do in the program. 
The project description \fxfatal{missing section reference}states that the program must be written in C\#, and therefore the program are written using this language. 
From the process analysis \fxfatal{missing section reference} we concluded that some people have a problem about packing their suitcase(s) to heavy and then getting fined when using some kind of transportation with limit on how much the luggage are allowed to weight \fxfatal{missing section reference}. This will be solved in the program in two ways. Firstly the program will pack the suitcases using weight optimization - it will, if more suitcases are availably to be packed, try to spread evenly between the suitcases. Secondly a function must be made that makes it possible for the user to set a max weight for the luggage. This can help on the problem that some traveling businesses have a max weight for luggage and will fine people carrying to heavy luggage \fxfatal{missing section reference}.
The program also needs to pack the best way according to the size of the suitcase(s) and item(s) so as many items as possible will be packed in the program, and that the program will not place any items outside the suitcase(s). This will help on the problem that some people sometimes has a problem with packing their suitcase(s) so everything will fit in the suitcase(s).
To make the program more user friendly a list of all the items to be packed should be included and be able to be edited so the user can get a better overview on that he/she is asking the program to pack.
To help the user through the process of packing the suitcase(s) the program will need the ability to show the user where to put the items in the suitcase(s). This is a very essential feature in the program since if the user are not able to see or get described where to put the items, the program will not in anyway help the problem. 











To help the user get started on the program, there will be a guide included in the program. The guide gives an explanation on how to use the program. The guide will be short and well formulated so the user with ease can read and understand the guide.
The project description \fxfatal{missing section reference}states that the program must be written in C\#.
The program itself needs to have a few features for it to solve the problem that is the focus of this project. The program needs to make sure that the weight is evenly distributed in the bags and that it does not exceed the bags weight limits. 

The program also needs to distribute the space of the bags to make sure that the program does not fill a bag more than there is physical room for.
When the user is on the trip the program needs to have a function that allows the user to edit the list of items that are in the bag, so that if the user buys some souvenirs or throws something away, the list of items will be updated and thereby a new way to pack the luggage is given.
The program also needs a function to help the user see where the items are placed in the luggage, so that the user can actually pack the suitcase(s).
The program will also have to check that the suitcases are below the limits set for weight and size.

There are some features that not are essential for the program to work but will improve the program. One of these features is to handle changeable shape of items e.g. a T-shirt or other forms of clothing. This makes the program able to pack more efficient. This means that to program can handle like solid, liquid and bendable shapes. But this may not be in the program at the start since this will be hard to develop and implement.
To help packing better and plan ahead, the program needs a list of different trip types which can help the user pack the luggage for a given type of trip.
Another nice feature to have is to allocate space for possible souvenirs the user might buys on the trip. These features means that the user does not need to check if there is room for the souvenirs before buying it.

\subsection{Targeted Features}
These are the essential features that the program will have.\newline

\textbf{Program language is C\#}:
The program need to written in C\# since the requirements for this project is that the program is written in C\#.

\textbf{Guide the user}:
The program will have a small "read me" file, or another form of guide, which will tell the user how to use the program.
\newline

\textbf{Distribute weight}:
The program must be able to distribute weight of items between multiple suitcases if there are more than one suitcase.
\newline

\textbf{Distribute space}:
The program also needs to distribute the items by space. The whole idea of the program, is that it should be able to tell the user how to pack the suitcase, and be able to tell if there is enough space for eventual souvenirs. Lastly it should inform the user how much space, if any, is left.
\fxfatal{Men den fortæller jo ikke hvor meget plads der er tilbage og om der kan være souvenirs!}
\newline

\textbf{On the road}:
The program will be able to tell you, while you are on the trip, if there is enough space for a souvenirs, if you input the dimensions and weight of that item. And if you want to remove an item from your luggage it can this as well.
\newline

\textbf{Baggage rules}:
The program will need to know basic baggage rules. For example the luggage must not weigh too much, and it must be below certain dimensions.
\fxfatal{Der er da intet med basis regler i programmet eller noget om dimensioner!}
\newline

\textbf{Structure of packing}:
When the user asks the program if an item will fit in the suitcase, the program will show exactly where in the suitcase the item will fit.
\newline

\textbf{Packing list}:
To make it easier for the user to know what will be packed an editable lists will be included.
\newline

\textbf{Save/Load function}:
To make the program easy to use on a trip, there is need for a function that can keep the data safe when the user closes the program.
\newline

\subsection{Optional Features}
These features as mentioned above, are additional features that might be able to be implemented later if possible.\newline

\textbf{Solid/liquid/bendable shapes}:
The program will also take in account that items might be bendable, and therefore fit in other ways than solid items. For instance a T-shirt can be folded in many ways and thus can be considered a liquid form as it can fit almost everywhere.
\newline

\textbf{Type of trip}
Depending on the nature of the trip different packing lists will be necessary because each trip might require different items.
\newline
