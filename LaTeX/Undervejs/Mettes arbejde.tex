\chapter{Conclusion}
In this the final chapter a conclusion will be made about how well the final program works and which requirements it meets.



In this section a thesis statement will be formulated which will be used to develop a
method to the problem of this project and to get a more precise problem to work with.
The method is used to try and solve the problem stated in the thesis statement.
The problem analyses shows that there are two general types of programs on the market.
One type is a form of a packing list that tells the user what they might want to pack,
Page 12 of 59
Group B130, Aalborg University P2 2. Problem Analysis
the other type of program is a packing program which packs containers for the user and
calculates the shipping cost to a designated location. These two types of programs only
fulfill parts of the criteria this project have put forward. With this in mind a thesis
statement have been formulated to help shape the solution for this projects problem.
• How can a program be developed which helps the customer through the progress of
packing one or more suitcases the most effective way by size and weight?
The meaning of this thesis statement is to research and develop a program that in
some way could handle the problem, but the consumer also plays a role in the problem.
Therefore the consumer must also be taken into account when it is being developed. The
reason for this is to make the program as user friendly as possible.

4.1.1 Targeted Features
These are the essential features that the program will have.
Program language is C#: The program need to written in C# since the requirements
for this project is that the program is written in C#.
Guide the user: The program will have a small "read me" file, or another form of
guide, which will tell the user how to use the program.
Distribute weight: The program must be able to distribute weight of items between
multiple suitcases if there are more than one suitcase.
Distribute space: The program also needs to distribute the items by space. The
whole idea of the program, is that it should be able to tell the user how to pack the suitcase,
and be able to tell if there is enough space for eventual souvenirs. Lastly it should
Fatal: Men den inform the user how much space, if any, is left.
fortæller jo ikke hvor
meget plads der er
tilbage og om der kan
være souvenirs!
On the road: The program will be able to tell you, while you are on the trip, if there
is enough space for a souvenirs, if you input the dimensions and weight of that item. And
if you want to remove an item from your luggage it can this as well.
Baggage rules: The program will need to know basic baggage rules. For example
Fatal: Der er da the luggage must not weigh too much, and it must be below certain dimensions.
med basis regler i
programmet eller noget
om dimensioner! Structure of packing: When the user asks the program if an item will fit in the
suitcase, the program will show exactly where in the suitcase the item will fit.
Packing list: To make it easier for the user to know what will be packed an editable
lists will be included.
Save/Load function: To make the program easy to use on a trip, there is need for
a function that can keep the data safe when the user closes the program.
4.1.2 Optional Features
These features as mentioned above, are additional features that might be able to be implemented
later if possible.
Solid/liquid/bendable shapes: The program will also take in account that items
Page 20 of 59
Group B130, Aalborg University P2 4. Design
might be bendable, and therefore fit in other ways than solid items. For instance a T-shirt
can be folded in many ways and thus can be considered a liquid form as it can fit almost
everywhere.
Type of trip Depending on the nature of the trip different packing lists will be necessary
because each trip might require different items.