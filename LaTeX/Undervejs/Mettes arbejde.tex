\chapter{Conclusion}
In this the final chapter a conclusion will be made about how well the final program works and which requirements it meets.

The program has been written in the programming language C\# and a user interface have been made to make it easier for the user to use the program. When opening the program the main window of the program will appear. Here the instructions on how to use the program will be so the user always will have them at hand. From the main window the user can either get informations about the creators of the program, load a previous saved list, manage the lists or start the packing. The program saves the lists on the computer at a place the user specifies. This means the program can be used at any time even without an internet connection. 

In the manage windows the user can see the list of added suitcases or items and be able to edit the list. The user will be able to add new items or suitcases to the list, delete items or suitcases or edit already made items or suitcases. The user will have to provide the program with the items dimensions and weight and when adding suitcases also the weight limit for the suitcase. From the manage windows the user can save the lists on the computer for later use. 

After done managing the lists the user are returned to the main window where a button will start the algorithm managing the packing. When the algorithm is done a window will appear showing a 3D-image of the suitcase with all the things to be packed inside. A list of the items will be on the right side and then an item is selected it will be highlighted in the 3D-image. The image can be dragged, rotated and zoomed so the user have the best possibilities to view the 3D-image as he/she wants.

So the program covers the targeted requirements for the product and therefore the program should solve the problem stated in the problem statement. It can be concluded that the program can pack one or more suitcases using weight distributing. It might be improved by a function that handles bendable shapes, but even throw the program is missing that feature it is able to pack one or more suitcases in a good efficient way. Therefore it is a good program that helps the user with the process of packing one or more suitcases while using weight distributing but it could be improved if more time was available. 