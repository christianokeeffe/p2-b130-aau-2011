\section{Solution suggestions} 	
This section will focus on different solutions to the problem about packing a suitcase. Each solutions difficulties and benefits will be explained. At the last subsection a solution will be selected and the choice explained. The solutions have been chosen from the mind map that can be seen in appendix ?.
\newline
\subsection{Application for smartphones}
This idea is to make an application that helps the user pack one or more suitcases. The user will need to put in the height, length, depth, weight and a name of the items to pack. The application will then calculate if all the items can be packed. All of these calculations means the application will need a server to make the calculations. This requires the customer to have access to the internet on the phone to be able to use the program. At the last step the application will show the user where to place each item. This means that the customer will have easy access to the program everywhere the user might bring the smartphone as long as it has access to the internet. On the other hand people without a smartphone would not be able to use the program. This solution requires learning how to write applications for smartphones in C# and therefore this solution requires modification to the time schedule. This solution is easy to bring everywhere because you rarely leave your phone. On the other hand people without a smartphone would not be able to use the program.
\newline
\subsection{Extension for an existing program}
A second solution could be to make an extension for another program that already exists. This extension should add the missing functions of the original program. It could either be an extension to the many packing lists. In that case the program should be able to use the information from the lists to calculate how to most efficiently pack all the items and afterwards show how to pack them. Another solution could be an extension to the e-Commerce Shipping calculator. This extension should be able to also pack smaller items like a suitcase and not a container. A problem with making an extension for another program is if the other program is written in a programming language not able to work with a program written in C#. Another problem is testing the program if the other program is not open source and therefore the company's permission is needed before the testing can begin. If the exiting program is not open source the company's permission and cooperation will be needed in the making of an extension of their product. The testing is important to be sure that the programs are able to work together as planned. On the other hand it is possible to make a program that focus more specifically on what is missing in the other program and therefore cover more of the important features. So this solution would need the company’s permission to make use of their code for testing if it is not open source, and it would have to be determined if the program would be able to work together with an extension written in C#. The solution would be able to cover more of the problem since the existing program would already have some of the features needed and the extension could cover even more.
\newline
\subsection{Program for the computer}
A third solution could be a program for the computer. The user will need to supply the program with the height, length, depth, weight and a name of the items to pack, and the program will then calculate if all the items can be packed and where in the suitcase. After the calculation the program will show where to put all the items. This will be by showing where the individual items need to be on a 2D or 3D figure of the suitcase. Making a program for the computer means the customer will need to bring his computer on the trip if he wants to use it on-the-road, but it will not need internet since the computer is strong enough to make all the calculation on its own. This solution requires some time to learn 3D editing if the display figure of the suitcase should be in 3D and this should be taken into account in the time schedule.  This solution makes it is easy to test as the code will be self written and it does not need internet to work, but on the other hand the customer needs to bring the computer if the user wants to use it on-the-road.
\newline
\subsection{Choise of solution}
When understanding the three solutions it is possible to determine, which of the solutions best solve the described problem. To determine the best solution it is needed to look at the pros and cons of each solution. The pro about the first solution is that it is easy for the user to bring the program. The cons are that you will need internet to use the program and this can be expensive on a vacation. Another con is that this solution requires time to learn of how to write an application for a smartphone.
The pro about the extension for another program is that since some code is already written there is more time to be more specific in what the original program is missing. On the other hand it can be a problem if the original program is written in a programming language that is not good at working together with a program written in C#. It is also a problem if the original program is not open source, because the company who own the program then needs to give permission to use their code. If the original program is not open source can it be a problem to get this permission without long negotiation with the company. The last solution is a program for the computer. A pro about this solution is that it does not need internet to run since the computer is strong enough to make the calculation on its own. Another pro is that since no code from others is needed there is no problem about different programming languages needing to work together. It is also easier to make the program exactly as needed since it is being made from scratch. A con is that some people do not want to bring their computer on a trip and can therefore not use the program while away. So when looking at all the solutions the choice is going to be the last solution about a program for the computer since it is the solution with the most pros and fewest cons. Also more time is not needed, which means there are more time to make the program and thereby more time to finish the program before the deadline.