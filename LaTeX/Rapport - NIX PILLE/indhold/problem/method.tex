\section{Method}
The structure of this project will be based on Aalborg PBL (problem based learning). The Aalborg PBL is a method whereby the learning process lies in investigating a problem and trying to develop a solution for the given problem.
The Aalborg PBL method also trains the students' ability to work together in a project group and gives them tools to handle the processes which go with working in a group.

The first stage of the project is the problem analysis in chapter \ref{chap:problem}, which purpose is to find and document that there is a problem to begin with. From the problem analysis a problem statement is formed and is used to produce a list of product requirements.
The requirements are then used to designing and developing a product that should solve the problem stated in the problem statement. The design will be described in chapter \ref{chap:design}.
The development will also have its own chapter where the program and how the different functions are made will be described. This can be seen in chapter \ref{chap:development}.

The testing phase will be described in chapter \ref{chap:testing}. The result of the testing will lead to improvements and a conclusion of the project. The conclusion will sum up the project and try to answer the problem statement. The conclusion can be seen in chapter \ref{chap:conclusion}. This is the main course of the project, when using the Aalborg PBL model.
This project form is used because it finds and documents a problem and then, through the work with the problem, gives an estimated solution to the problem.

To document the problem, a lot of information is needed. The information is found through different sources such as; books, articles, websites, etc. When using information found through the Internet or other sources it is important to evaluate the used sources.
This is done to filter out unreliable sources and thereby achieve a better and more trustworthy project.
This process of evaluation is also known as source criticism and is generally used when using other peoples materials as documentation. Therefore, it is also a relevant method when using sources in the project work.