\section{Sketches before programming of GUI}
\label{sec:sketches}
%Til retter: Vær OBS på første afsnit, er skrevet til
This section will describe the first thoughts of the GUI (Graphical User Interface). This is not how the final GUI is designed, but the first sketches.
Designing the program required some sketches and the first sketch looked like Figure \ref{fig:skitse1}.
It shows some text boxes where the user must type the dimensions of the suitcase, so the program knows how much space it has to deal with, and that the measurements are in the metric unit system.

\figur{0.9}{Skitse1.png}{Initial sketch of the program}{fig:skitse1}

It has been discussed whether there should be a 3D image or a 2D image of the suitcase, where the user can see where the items are placed. It was decided that the best way to show the packing, is by a rotative 3D image, if possible.
The point with the boxes was that it was there the measurements of a specific item would be shown.

\figur{0.9}{Skitse3a.png}{Second sketch, now with added control buttons}{fig:skitse3a}

Figure \ref{fig:skitse3a} shows how the idea were before the coding of the program began. The idea was that the suitcase would be shown and then one item would be drawn at a time. The user would then be able to see where each item should be put in the suitcase according to the program.

The difference between this and the previous sketch is that now there is a user-generated list with all the items. It is also now possible for the user to add names and amounts of specific items. The point was that the program would continuously update the 3D-viewer while each item was added to the list.

Figure \ref{fig:skitse4} is a look at how the packing list should look like, and what functions it should contain.

\figur{0.9}{Skitse4.png}{Sketch of the part where the user can add items}{fig:skitse4}

At last the final sketch for the GUI was decided. The 3D image viewer was replaced with the list of added items in the manage items and manage suitcases, because it gave a better overview of what the user already had added to the program. This can be seen on Figure \ref{fig:skitse1done} 

\figur{0.9}{1skitsedone.jpg}{Sketch of the main window}{fig:skitse1done}

What was changed from the first sketch of the main window was that more buttons were added. The two new buttons made it possible to load saved lists and start packing without needing to make changes to anything in the lists or making a new list.

\figur{0.9}{2skitsedone.jpg}{Updated version of the window where the user can manage items}{fig:skitse2done}

Figure \ref{fig:skitse2done} is the sketch of the window that opens when the "Manage Items"-button was clicked. Here the 3D viewer has been replaced with a list of the items added in the program. The buttons are tools for adding, deleting or managing the items on the list while the "Save item list"-button lets the user save the list so it can be used another time and returns them to the main window.

\figur{0.9}{3skitsedone.jpg}{Window where the user can manage suitcases}{fig:skitse3done}

The window seen on Figure \ref{fig:skitse3done} opens when the "Manage Suitcase"-button was clicked. Like in the "Manage Items" window the 3D viewer has been replaced with a list of all the added suitcases. When it comes to functionality the buttons does the same as those in the "Manage Items" window just with suitcases. Clicking the "Save Suitcase List"-button will let the user save the list and will return them to the main window.

\figur{0.9}{4skitsedone.jpg}{Sketch of the window that appear if the user clicks the "New packing list"-button}{fig:skitse4done}

When clicking the "New packing list"-button the window seen on Figure \ref{fig:skitse4done} will be shown. Here you are able to add the suitcases, you want the program to pack. On the left side in the window the program will show a 3D picture of the suitcase. The "Add items"-button will send you to the next window.

\figur{0.9}{5skitsedone.jpg}{Window opened when the user adds new items}{fig:skitse5done}

Figure \ref{fig:skitse5done} shows how the user can add the items they want packed to a list. After the user have added all the items the "View list"-button will show a list with all the items and afterwards return the user to the main window.

\figur{0.9}{6skitsedone.jpg}{Sketch of a window that shows the saved lists}{fig:skitse6done}

Figure \ref{fig:skitse6done} is the window that will be shown when the "Load saved list"-button was clicked. Here the user would be able to load a saved list and delete lists they would not want to save anymore. The "Back"-button will return them to the main window.

\figur{0.9}{7askitsedone.jpg}{Sketch of the 3D-viewer of the program}{fig:skitse7adone}

Figure \ref{fig:skitse7adone} is one of the suggestions on how the window showing where the items should be packed, could look like. Here the 3D viewer show the position of every item, one item at a time. The two buttons are used to go back and forth through the items.

\figur{0.9}{7bskitse.jpg}{Sketch of the 3D-viewer with instructions added}{fig:skitse7bdone}

Figure \ref{fig:skitse7bdone} is the second suggestion on how the window should look like. Here the 3D viewer is smaller but on the other hand has a list of instructions on the side to better help guide the user. The two buttons are used to go back and forth through the items.

\figur{0.4}{8skitse.jpg}{Sketch of the message box that will be shown when the items have been packed}{fig:skitse8}

The message box seen on Figure \ref{fig:skitse8} will be shown when all the items have been packed. The "Pack again"-button will show where to put all the things again. The "Make new list" will return the user to the main window where they can start making a new list, load a saved list or manage the already saved list. Most of the sketches do not look like the program what so ever, this is because the sketches where made before the program, and when the program was developed, the GUI was adapted to be more logical because of some good feed back from the tests.
