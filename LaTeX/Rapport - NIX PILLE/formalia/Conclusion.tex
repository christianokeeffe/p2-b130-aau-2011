The final chapter of the report is a conclusion on the whole project and a conclusion on the problem statement.

To make a conclusion on this project it is necessary to take a closer look on the problem statement, see section \ref{sec:thesis}. To meet and answer the problem statement, the program must be able to pack one or more suitcases by volume and weight in an effective way. It is also required that the program must present the result to the user in some way before it actually can help the user.

The program is capable of packing one or more suitcases with items while using the volume as effectively within the programs reach. When mentioning the programs reach it is that the program handles the items as boxes instead as their real shape.
The program does not pack as effectively as possible because that will include that the program treats items as different objects and not just boxes. The program also distributes the weight evenly among the suitcases if there is more than one and checks that the suitcases do not exceed a given limit. The program also has a user interface that fulfills the role as provider of the result.

The program gives a packing solution but not as effectively as possible and thereby the program itself does not give a full answer to the problem statement. The answer should be found through experience and knowledge gained in this project. The reason for this is that with new knowledge it will be possible to take this program and improve it so program better handles the items that should be packed. So for the program to be able to pack more effectively it must be better at handling all kinds of shapes and be able to place them compared to each other.

So it can be concluded that problem statement has been answered through the work with the final product. The problem statement is not answered through the product itself. The reason is that the program only gives the most effective packing solution within the criterion that items is handle as boxes.

The program also solves the problem stated in the project suggestion, about making a program able to help with planning how to pack a suitcase (see appendix \ref{chap:forslag}. It can be concluded that the program can pack one or more suitcases using weight distributing. It might be improved by a function that handles bendable shapes, but even though the program is missing that feature it is able to pack one or more suitcases in a good and efficient way. Therefore it is a sufficient program that helps the user with the process of packing one or more suitcases while using weight distributing but it could be improved if more time were available.

%The program has been written in the programming language C\# and a user interface has been made to make it easier for the user to use the program. When opening the program the main window of the program will appear. Here the instructions on how to use the program will be shown so the user always have them at hand. From the main window the user can either get informations about the creators of the program, load a previous saved list, manage the lists or start the packing. The program saves the lists on the computer a user specified path. This means the program can be used at any time even without an Internet connection. 

%In the manage windows the user can see the list of added suitcases or items and be able to edit the list. The user will be able to add new items or suitcases to the list, delete items or suitcases or edit already made items or suitcases. The user will have to provide the program with the items dimensions and weight and when adding suitcases also the weight limit for the suitcase. From the manage windows the user can save the lists on the computer for later use. 

%After the users are done managing the lists the user is returned to the main window where a button will start the algorithm managing the packing. When the algorithm is done a window will appear showing a 3D-image of the suitcase with all the things to be packed inside. A list of the items will be on the right side and when an item is selected it will be highlighted in the 3D-image. The image can be dragged, rotated and zoomed so the user have the best possibilities to view the 3D-image as he/she wants.

%The program covers the targeted requirements for the product and therefore the program solves the problem stated in the problem statement.