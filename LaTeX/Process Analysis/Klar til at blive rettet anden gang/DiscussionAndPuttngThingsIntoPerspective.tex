\chapter{Discussion and Putting Things into Perspective}
In this chapter we will discuss and perspective over this project and the work regarding it.
The working time of the project was from 9:20 to 16:15 every day given that no lecture was scheduled that day. By doing this there was a high productivity in the group, but still there was time for fun. To improve the work, the option to work from home from 14.00 was removed when we went behind schedule.
We did this because working in the group room meant a higher productivity in the group. One of the good features for a group is that we are social and friendly with each others because it enhances and improved the group spirit. This can be useful for work in the future - that you can work with other people and be part of a group.

We nearly had a meeting with the supervisor every week to make sure that we where on the right track with the project all the time. There was a single time when a supervisor meeting was pointless because the week before, the only thing that had been made was on the program. The overview of our project was easy to keep since the structure of the report was well organized. The schedule was well made which made it easy to keep track of the progress of the project. In this project there was a few roles in the group, the roles that were, was the ones that we thought where necessary to ensure optimal group work. This time there was only made a summary for the meetings with the supervisor and not for the morning meetings since it was not used that much as in the previous project. 

Compared to our P1 project, we had to worked somewhat harder, with the result being a better report and a good product. Our working progress has been nearly the same, but since we got behind the schedule for the project, we had to work a little harder. But the method for project work makes the process easier to comprehend and thus making it good for future projects.