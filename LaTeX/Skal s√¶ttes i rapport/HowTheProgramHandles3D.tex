\section{How the Program Handles 3D}
\label{sec:3DHandler}
The 3D in this program is handled by objects called Shape3D. A Shape3D object contains an array of Polygons objects. A Polygon object contains an array of Vector objects and a Pen. A Vector object contains three integers: x, y and z, which are the coordinates of the Vector. 

Instead of programming the whole 3D part ourselves, an existing solution was found and developed further. The original solution \citep{orgcode} was only a demonstration on how to draw a cube, an axis and a more advanced polygon using C\#. This solution was further developed to support drawing cubes of different sizes. Along the development, unnecessary features like drawing an axis was removed. The interface of the 3D was also improved, so that no GUI elements where created directly in the code that handled the 3D.

\subsection{Shape3D Class}
A Shape3D object contains four methods: Draw, Rotate, ScaleSize and RePaint.

The Draw method is really just a "foreach" loop which calls the Draw method in the Polygon object on all Polygon objects within the Shape3D. The Draw method in the Polygon object will be explained in \fxfatal{indsaet en ref}.

The Rotate method in the Shape3D object is used to rotate the object. The method takes one input parameter, which is a RotateMatrix called "mat". Next the method creates a new Polygon array newP, and runs through all Polygon objects in the Shape3D. The method then calls the Rotate method within the Polygon object on the current polygon, and inserts the results into the Polygon array newP. The method also inserts the current Polygons Pen into the newP Polygons Pen. The function can be seen on Listing \ref{lst:Shape3DRotate}.
\kode{Function in Shape3D} {Shape3DRotate}{Shape3DRotate.txt}.

The ScaleSize method in the Shape3D object is almost the same as the Rotate method in Shape3D. The only difference is that instead of calling the Rotate method in the Polygon object, the ScaleSize method calls the ScaleSize method in the Polygon object.

The RePaint method in the Shape3D object is used to mark one item in the suitcase(one polygon), with a thicker line than the others. In the program this is used when the user clicks on an item in the list of items to the right of the drawing of the suitcase. If the user clicks on an item in the list, the program will mark that item with thicker lines. This function is also used when the 3D viewer is opened - a small cube with dimensions 0.5 x 0.5 x 0.5 is drawn in the suitcase's 0,0,0 point, so the user knows where the 0,0,0 point is. This cube is also marked with a thicker width than the items.

The method takes two integers as input parameters: "num" and "col\_w". The "num" input is the item that has been selected, and the "col\_w" is how wide the line must be. The method initiates a new Polygon array, newP, where all the polygons will be stored. The method also contains three counters. One that counts the elements in the polygon, this counter is called "element\_counter". Here an element consists of four sides. Another counter called "counts" is counted up when a polygon is made thicker, this is counter is called "counter". The last counter is counted up whether or not a polygon is made thicker. This counter is called "count" which keeps track of the Polygon array newP.

First the method checks if the color of the current polygon is red. This only happens if the current polygon is the cube representing the 0,0,0 point. If the color is red, the width of the polygon is set to 2, and the counter "counter" is counted up. The method now checks whether or not "counter" is 4. This occurs if all four sides of a polygon is made thick. If this is the case, the counter "counter" is set to 0, and the element counter is counted up, implying that the method now should check the next element in the Shape3D object. This can be seen on Listing \ref{lst:TrollDetected}.

\kode{Function that makes an item thicker}{TrollDetected}{makephatfunction.txt}

At the end the method returns a new Shape3D object, newP, which replaces the old Shape3D, and is drawn.

\subsection{Polygon Class} 
The Polygon class contains five methods: Draw, X2D, Y2D, Rotate and ScaleSize. The essential method in this class is the Draw method. This is the method that actually draws the suitcase and items on the form. The Draw method takes two input parameters; a Graphics object "g", and a Pen "MyPen". The Pen is where the color and width of the lines are stored. Before explaining how the Draw method actually works, a closer look will be taken at where it is first called.

Every time the method 'Invalidate' is called, the Draw method in the current Shape3D object is also called. The 'Invalidate' method is called every time the program needs to refresh the drawing of the suitcase and items. For example when the user changes suitcase in the drop down element, the program needs to redraw the image, so that the items in the new suitcase are drawn instead. 

In the frm3DViewer, the Draw method is called inside the method MyPaint, which takes two input parameters: an 'object' "sender", and an 'PaintEventArgs' "e". The method then calls the Draw method in the Shape3D, with the parameter e.Graphics. The Draw method in Shape3D receives the 'Graphics' "g". The method then runs trough each Polygon in the Shape3D and calls a Draw method in the Polygon class on each of the Polygons. It is now clear that the Draw method is called on each Polygon in a Shape3D.

The Draw method in the Polygon class begins with a "for" loop which runs through the length of the Polygon. The length of a Polygon is defined by the combined length of the Vectors within the Polygon. Inside the "for" loop, two lines are drawn with the built in method DrawLine in the Graphics class. The DrawLine method takes three input parameters, a Pen, and two points. But the Polygons do not contain any points. Therefore the Vector must be converted to points before it is drawn. The Draw method in the Polygon class can be seen on listing \ref{lst:pokpok}.
\kode{Draw function in the Polygon Class}{pokpok}{polygondraw.txt.txt.txt.txt.txt.txt}

The Rotate method in the Polygon class basically receives a matrix, which contains how the Polygon should be rotated. The method then run through each vector in the Polygon calling the Rotate method in the Vector class on each Vector with the rotate matrix as input. When this is done a new Polygon is returned. This is the rotated Polygon.

The ScaleSize method in the Polygon class is used to scale the polygon with a factor. This is used to zoom the drawing in and out. The method receives a float as input, runs through every Vector in the Polygon and calls the ScaleSize within the Vector class on every Vector, parsing the float as parameter.  When this is done a new Polygon is returned. This is the scaled Polygon. The ScaleSize method in the Polygon class can be seen on Listing \ref{lst:pokpokpok}.

\kode{ScaleSize function in the Polygon Class}{pokpokpok}{PolygonScaleSize.txt}
