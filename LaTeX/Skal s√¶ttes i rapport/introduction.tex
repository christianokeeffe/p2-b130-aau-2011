This project is based on the subject "pack the suitcase" with a focus on helping users to pack a suitcase properly, with an eye on weight optimization. From this project a product should emerge that will try and deal with a problem within the subject's field. When the product is done, it should be tested and based on the test it will be possible to evaluate the product's capability at solving the problem.

It is a requirement that the program is made in the program language C\#. It is not a requirement that the program has a graphical user interface (GUI), but it has been chosen to do anyway.
A part of the project is to document a problem and analyze it, to find the reason behind the problem. When the analysis is done, the product should be designed to handle the problem and thus solve it.

This is the essential of problem-based learning: To find a problem and then solve it and hopefully gain some knowledge in the process.
For this to be an effective model all the group members must participate in the process and acquire knowledge by working with the project.

This project was in the project catalog, can seen in appendix \ref{chap:forslag}, which was made by the semester coordinator, Hans Hüttel, and this project was chosen after a long discussion based on the expectations in the group and the given time.