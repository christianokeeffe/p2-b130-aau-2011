\section{Specification Requirements}
\label{sec:Spec}
Through the problem analysis it has been documented that there are some strict rules regarding some forms of public transportation when going on vacation. Based on this research a list of features have been composed, that the program must fulfil to meet the base requirements.
Furthermore another list have also been made composed of some additional features that would make the program better and more user friendly. They are not needed for the base requirements, but rather as improvements to further make the program ideal for the user.

For the user to better get started on the program there will be a guide that come with the program. The guide gives an explanation on how to use the program. The guide will be short and well formulated so the user with ease can read and understand the guide.
The project description states that the program language must be written in C\#.
The program itself needs to have a few features for it to solve the problem that is the focus of this project. The program needs to make sure that the weight is evenly distributed in the bags and that it does not exceed the bags weight limits. The program also needs to distribute the space of the bags to make sure that the program does not fill a bag more than there is physical room for.
When the user is on the trip the program needs to have a function that allows the user to edit the list over items that are in the bag so if the user buys some souvenirs or throws something away, the list of items will be updated and thereby a new way to pack the luggage.
The program will need a function to help the user see where the items are placed in the luggage.
The program will also have to check that the suitcases are below the limits set for weight and size.


There are some features that not are essential for the program to work but will improve the program. One of these features is to handle changeable shape of items e.g. a T-shirt or other forms of clothing. This makes the program able to pack more efficient. This means that to program can handle like solid, liquid and bendable shapes. But this may not be in the program at the start since this will be hard to develop and implement.
To better help packing and planing ahead the program needs a list of different trip types that can help the user with packing the luggage for a given type of trip.
Another nice feature to have is to save space for possible souvenirs the user might buys on the trip. These features means that the user does not need to check if there is room for the souvenirs before buying it.

\subsection{Targeted Features}
These are the essential features that the program will have.\newline

\textbf{Program language is C\#}:
The program need to written in C\# since the requirements for this project is that the program need to written in C\#.

\textbf{Guide the user}:
The program will have a little "readme" file, or other form of guide, that will tell the customer how to use the program.
\newline

\textbf{Distribute weight}:
The program must be able to distribute weight of items between multiple suitcases if there are more than one suitcase.
\newline

\textbf{Distribute space}:
The program also needs to distribute the items by space. The whole idea of the program, is that it should be able to tell the user how to pack the suitcase, and be able to tell if there is enough space for eventual souvenirs. Lastly it should inform the user how much space, if any, is left.
\newline

\textbf{On the road}:
The program will be able to tell you, while you are on the trip, if there is enough space for a souvenirs, if you input the dimensions and weight of that item. And if you what to remove a item from your luggage the it can this as well.
\newline

\textbf{Baggage rules}:
The program will need to know basic baggage rules. For example the luggage must not weigh too much, and it must be below certain dimensions.
\newline

\textbf{Structure of packing}:
When the user asks the program if an item will fit in the suitcase, the program will show exactly where in the suitcase the item will fit.
\newline

\textbf{Packing list}:
To make it easier for the user to know what will be packed an editable lists will be included depending on the type of trip.
\newline

\textbf{save/load function}:
To make the program easy to reuse or when to use on a trip there is need for a function that can keep the data safe when the user close the program.
\newline

\subsection{Optional Features}
These features as mentioned above, are additional features that might be able to be implemented later if possible.\newline

\textbf{Solid/liquid/bendable shapes}:
The program will also take in account that items might be bendable, and therefore fit in other ways than solid items. For instance a T-shirt can be folded in many ways and thus can be considered a liquid form as it can fit almost everywhere.
\newline

\textbf{Type of trip}
Depending on the nature of the trip different packing lists will be necessary because each trip might require different items.
\newline
