\section{GUI Description}

The main window of the program looks like this.

\figur{}{..\..\GUI\Screenshots\Mainwindow.png}

In the main window there are instructions on how to use the program, there is an "About" button which tells who we are and what we do. There are the "Manage Items" and "Manage Suitcases" buttons where the user can add/edit items/suitcases. The button "Load Saved List" can be used if the user previously has made a list and wants to add/edit some items or a suitcase. The "Start Packing" button starts the program's algorithm and packs the suitcase(s). The progress bar is associated with the "Start Packing" button and starts when the user presses "Start Packing".

\figur{}{..\..\GUI\Screenshots\ManageItems.png}
The Manage Items menu is where the user can add and/or edit items.

\figur{}{..\..\GUI\Screenshots\ManageSuitcase.png}
Add/edit suitcases in the Manage Suitcase menu.

\figur{}{..\..\GUI\Screenshots\About.png}
The About button which tells who the programmers are, and it shows when the program was made.

\figur{}{..\..\GUI\Screenshots\AddItem.png}
When the user presses the "Add Item" button in Manage Items, this form shows. In the form there is a text box, where the user types the name of the current item. There are 5 other text boxes which are for the length, width, height, weight, and amount of that item.

\figur{}{..\..\GUI\Screenshots\AddSuitcase.png}
When the user presses the "Add Suitcase" button in Manage Suitcases, the user can add and/or edit a suitcase's data. The length, width, height, weight, and the maximum weight of the suitcase. 

\figur{}{..\..\GUI\Screenshots\EditItem.png}
If the user wants to edit an item, he/she can press the "Edit Item" button in Manage Items, and this form shows. In the form there are 6 text boxes for each input parameter, and a button saying "Edit", which saves the changes the user has made and closes the form.

\figur{}{..\..\GUI\Screenshots\EditSuitcase.png}
In "Manage Suitcase", there is a button called "Edit Suitcase". It allows the user to change the data of a suitcase, if e.g. the measurements are wrong, or the user wants to use another suitcase, which have not the same measurements.

\figur{}{..\..\GUI\Screenshots\LoadSavedLists.png}
If the user already has used the program before and has saved an item list and a suitcase list, both can be loaded here.

\subsection{3D viewer}
The 3D viewer shows how the program have packed the different items in the suitcases.

\figur{}{..\..\GUI\Screenshots\frm3DViewer.png}

This is the first thing the user will see when the user has started packing. It shows how the items are placed in the suitcase.
The image can be dragged, moved, and zoomed with the mouse, as seen below. When the user clicks on an item in the list on the right side, the marked item will be highlighted in the image. Below the list are the xyz-points to see where the item's supposed to be placed, and the current suitcase's weight. 

\figur{}{..\..\GUI\Screenshots\frm3DViewer2.png}

On the left side of the window are two buttons, zoom in and zoom out. There is a check box on the lower left side called "Zoom limit". It sets limit for how close and how far the user can zoom the image. The buttons are made if the user has not got a mouse with a scrolling wheel or is on laptop. The track bar on the left is a tool to adjust the speed when the user rotates the image. The reset button resets the track bar. Below the reset button is a drop-down list with the suitcases the user has packed items into.

The reason why it looks like it does is because it gives a good overview with the item list on the right, the 3D-image in the middle, and the image options (zoom in, zoom out, reset etc.) on the left.

In the 3D-image a small box in point (0;0;0) is made to show where it is, so the user easier can navigate through the items.