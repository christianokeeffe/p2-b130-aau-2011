\section{Specification Requirements}
\label{sec:Spec}
Through the problem analysis it has been documented that there are some strict rules regarding some forms of public transportation when going on vacation. Based on this research a list of features have been composed, that the program must for fill to meet the base requirement to help the user.
Furthermore another list have also been made composed of some additional features that would make the program better and more user friendly. They are not needed for the base requirement, but rather as an improvements to further make the program ideal for the user.

\subsection{Targeted Features}
These are the essential features that the program will have.\newline

\textbf{Guide the user}:
The program will have a little "readme" file, or other form of guide, that will tell the customer how to use the program.
\newline

\textbf{Distribute weight}:
The program must be able to distribute weight of items evenly in each individual suitcase and if needed spread out in multiple suitcases.
\newline

\textbf{Distribute space}:
The program also needs to distribute the items by space. The whole idea of the program, is that it should be able to tell the user how to pack the suitcase, and be able to tell if there is enough space for eventual souvenirs. Lastly it should inform the user how much space, if any, is left.
\newline

\textbf{On the road}:
The program will be able to tell you, while you are on the trip, if there is enough space for a souvenirs, if you input the dimensions and weight of that item.
\newline

\textbf{Baggage rules}:
The program will need to know basic baggage rules. For example the luggage must not weigh too much, and it must be below certain dimensions.
\newline

\textbf{Where in the suitcase}:
When the user asks the program if an item will fit in the suitcase, the program will show exactly where in the suitcase the item will fit.
\newline

\textbf{Packing list}:
To make it easier for the user to know what will be packed an editable lists will be included depending on the type of trip.
\newline

\subsection{Optional Features}
These features as mentioned above, are additional features that might be able to be implemented later if possible.\newline

\textbf{Solid/liquid/bendable shapes}:
The program will also take in account that items might be bendable, and therefore fit in other ways than solid items. For instance a T-shirt can be folded in many ways and thus can be considered a liquid form as it can fit almost everywhere.
\newline

\textbf{Type of trip}
Depending on the nature of the trip different packing lists will be necessary because each trip might require different items.
\newline

\textbf{Number of people}
Usually a trip is done with more then one person, so more suitcases might be available to distribute items between.
\newline

\textbf{Account for the trips length}:
If a long trip is planned, the program can take in account that the user might need more space for souvenirs, so the user do not need to check on the trip if there is room and weight for every souvenirs in the luggage and decide if the souvenir can come in the luggage without exceed the weight limit.
\newline