\section{Luggage rules}
This section will focus on the general rules regarding luggage when going abroad, whether by plane, train or cruise ship.

\subsection{Charter trips on air planes}
Given below is the rules for varies items you could bring on a plane.
\newline
\newline
\underline{Checked-in luggage}\newline
Check-in luggage is the luggage that will go in the planes cargo hold.
\newline 
Items not allowed:
\begin{itemize}
\item Explosives, including detonators, fuses, grenades, mines and explosive compounds
\item Gasses, Propane, butane
\item Flammable liquids, including petrol, methanol
\item Flammable solid matter and reactive, including magnesium, matches, fireworks, flares
\item Oxidising and oxidised compounds and organic peroxides, including bleach, auto repair-kits.
\item Toxic or contagious compounds, including rat poison, infected blood.
\item Radioactive materials, including medical isotopes and isotopes for industrial use
\item Corrosive compounds, including quicksilver, car batteries.
\item Compounds from combustible systems, which have contained fuel.
\end{itemize}
Due to the volatile or dangerous nature of the items listed above they have been deemed unsafe and thus not allowed on the plane without explicit permission from the airport.\newline
\newline
\underline{Carry on}\newline
Carry on luggage is what the passenger is allowed to bring aboard in the cabin.
\newline
Approved items:
\begin{itemize}
\item Liquids, perfume, gel and spray – max. 100 ml – equal to one decilitre pr. container
\item You are only allow to bring these containers (bottles, cans, tubes and, so on), if they are contained in a transparent plastic bag, which have to be closed (1 litre bag per passenger).
\item The bag have to be resealable.
\item Past security, wares can be purchased (including spirits, perfume and other liquids). Wares are handed out in sealed bags, these bags may only be opened after the final destination have been reached.
\item It is now a requirement that you take off your overcoat, take laptops and other larger electronic devices out of the bag before the security check-in.\\
\citep{Prohibited_luggage}
\end{itemize}

\subsection{Rules on trains}

There are different rules depending on which train company you are using. 
\newline
The Danish train company, DSB, have very few rules regarding the luggage you are allowed to bring with you. 
\newline\newline
The only other rule is that your luggage need to be able to lie on the luggage rack or under the seat and not be bothering or putting any other person on the train in danger \citep{rulestrain}.
\newline\newline
Another example could be Indian Railways where the luggage is allowed to have different weight depending on which class you are on. They have no other rules regarding luggage \citep{idianrules}.

\subsection{Rules on cruise ships}
On board a cruise ship the "rules" are not really rules more like guidelines as they encourage the passengers to not exceed the limits. Furthermore the passengers luggage should be kept in their cabin during the trip\citep{Cruise}.

\subsection*{Luggage table}
This table displays the summary of the different rules from above.

\begin{table}[H]
\begin{center}
\begin{tabular}{| c | c | c |}
\hline
Type of luggage &  Dimension limit & Weight limits \\ \hline
Check-in luggage(Airplane) & 158 cm (Height+Width+Depth) & 20-23 kg \\ \hline
Extra luggage(Airplane) & 158-277 cm (Height+Width+Depth) &  20-45 kg \\ \hline
Carry on(Airplane) & Height 50-55 cm + Width 40 cm + Depth 18-25 cm & 5-8 kg \\ \hline
Luggage(Train) & 100 x 60 x 30 cm & Within reason \\ \hline
Check-in luggage (Cruise) & 75 x 50 x 29 cm & 30 kg \\ \hline
Hand Luggage (Cruise) & 55 x 35 x 25 cm & Within reason \\ \hline
\end{tabular}
\end{center}
\end{table}