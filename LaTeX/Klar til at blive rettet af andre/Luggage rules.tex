\section{Luggage rules}

\subsection{Charter trips on air planes}

Check-in luggage
\newline
Dimension limit\indent\indent\indent 158 cm (Height+Width+Depth)
\newline
Weight limit\indent\indent\indent\indent 1 suitcase x 20-23 kg
\newline\newline
Extra luggage
\newline
Dimensions\indent\indent\indent\indent 158-277 cm (Height+Width+Depth)
\newline
Weight \indent\indent\indent \indent\indent  20-45 kg
\newline\newline
Items usually not allowed in the checked-in luggage:
\newline
Explosives, corrosive and flammable compounds e.g. gas, methylated spirit, paint and the like  
\newline
Oxygenated, toxic and radioactive compounds 
\newline
Flammable gases 
\newline
Magnetic materials 
\newline
Fireworks 
\newline
Sedatives 
\newline
Bleach 
\newline
Paint
\newline\newline
Carry on
Approved items: Liquids, perfume, gel and spray – max. 100 millilitres – equal to one decilitre pr. container
You are only allow to bring these containers (bottles, cans, tubes and so on), if they are contained in a transparent plastic bag, which have to be closed (1 litre bag per passenger)
The bag have to be resealable.
\newline
Past security, wares can be purchased (including spirits, perfume and other liquids). Wares are handed out in sealed bags, these bags may only be opened after the final destination have been reached.
\newline
It is now a requirement that you take off your overcoat, take laptops and other larger electronic devices out of the bag before the security check-in.
\newline\newline
Dimension limit \indent\indent\indent	Height 50-55 cm + Width 40 cm + Depth 18-25 cm
\newline
Weight limits\indent\indent\indent\indent		5-8 kg

\subsection{Rules on trains}

There are different rules depending on which train company you are using. 
\newline
Look at the Danish train company, DSB, for example. They have very few rules regarding the luggage you are allowed to bring with you. 
\newline\newline
Dimensions\indent\indent\indent\indent 100 x 60 x 30 cm 
\newline\newline
The only other rule is that your luggage need to be able to lie on the luggage rack or under the seat and not be bothering or putting any other person on the train in danger. (http://www.dsb.dk/kundeservice/service-i-toget/bagage/)
\newline\newline
Another example could be Indian Railways where the luggage is allowed to have different weight depending on which class you are on. They have no other rules regarding luggage. (http://www.trainenquiry.com/StaticContent/Railway\_Amnities/E\_R/LUGGAGE.aspx)