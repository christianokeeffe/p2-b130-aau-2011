\section{Bin Packing Problem}
\label{sec:binpacking}
Bin Packing problems is a combinatorial NP-hard problem. The problem consists of fitting objects of different sizes into bins of identical sizes \ref{appofdismath}. This could for example be fitting various packages into shipping containers. There are various approaches to solve the bin packing problem. Some of the popular methods will be described in the following. 

\subsection{First fit (FF)}
The first fit algorithm creates a list of the objects needed to be fitted in bins. It then runs through the list, checking if an item can fit in each bin: If it cannot fit in the first bin, it will check if it can fit in the second bin and so on. If it does not fit in any bins, it opens a new bin, and fits the object there. 

\subsection{Best fit (BF)}
The best fit algorithm is the same as the first fit algorithm, except that before an object is packed, the algorithm checks each open bin, where the object fit. It will then place the object in the bin which will have the least space left when the object is packed. 

\subsection{First fit Decreasing(FFD)}
The above algorithms are very ineffective because the biggest objects might be placed at the end of the list, and thus be packed in the end, where it is more effective to pack these large objects at first.
The first fit decreasing algorithms takes this in account and sorts the list before attempting to pack the items. This way the biggest items will be packed at first.

\subsection{Best fit Decreasing(BFD)}
Again this is the same as the best fit algorithm, but with the list being sorted before attempting to pack the objects.

\subsection{Round up}
It seems that it is smarter to sort the lists before attempting to pack objects in bins. This way bigger objects are packed first, and the smaller objects can then be fitted around the bigger objects. However in some situations it is necessary to use unsorted lists. For example in a factory with continuous production, it is never possible to have the complete list of objects, and thus never possible to sort the list.
