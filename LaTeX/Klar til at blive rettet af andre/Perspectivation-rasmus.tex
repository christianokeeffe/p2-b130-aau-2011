\subsection{Perspectivation}
When working with problem solving it is also important to look at how the user will use and interact with the solution. Through these assumptions a better solution can be form that are more user friendly. Therefore it is a good thing to know who that are going to use the solution and that way design it a more thoughtful way.
The purpose in this project was to make a program that provided the user with a packing solution and help the users to packing a suitcase more effectively. The program should also control that the weight of the suitcase does not exceed the given limits by the user. The program is made in visual studio that provides the developer with a lot of good tools to make an user interface for the user of the program.
This makes the program a lot more user friendly because the user easier can interact with the program. This means that it will be a lot easier for the user to operate the program and thereby the program have bigger chance to sell if it were the intention.
The user interface also means that the user do not need understand the program before they can use it.
The program in this project also supports mouse control, that means that user can use his/her mouse to navigate around in the program. Because of the mouse support and the good interface the user have less chance to fail in using the program but by implementing an interface there is a chance for new bugs and misunderstandings in the formulations in the interface.
Therefore it is important to make a clean and user friendly interface that only the necessary buttons and well formulated text that help the user by information about what they about to do and what unit the data is in.

To prepare the program and try to find eventually problems the program have undergone some tests, where people that not have worked with the project have tried the program. Their experiences from the test are then collected and use to evaluate the program. The evaluation then led to changes that would make the program better. After the changes were made the program were tested again with the changes and the similar processing of the test result as the earlier test.
This process of repeated testing is a good procedure to develop a working program because the potential user gets to use it and experiences are made in context to the use. This experiences are then used to make the program more suited for the target audience. The downside to this procedure is that it takes a lot of man hours to do the testing and make the charges. Therefore testing is good but expensive tool to use.
It should be taken into consideration to test the program, because it is important that the program is free of the worst bugs when released. Program bugs on release have a negative influence on the programs sale and might scare of potential purchaser.

A side effect from making the program in Visual Studio is that the program only can run on Windows Vista or Windows 7 (Microsofts operating system). The reason is that the program uses the Microsoft.NET framework that only are supported by Microsofts operating systems.
This means that it is important to take a moment before developing and make a through regarding what platform or better yet, if it should be able to run cross platforms.
These different experience made in this project can be use in other upcoming project.