\section{File Serialize function}
To save the data in the program a function in C# called File Serialize is used. The function is called when the data from the program should be saved or loaded. The function lies in the file "FileSerializer.cs", which is taken from the website "www.codeproject.com" from the "how to" "Custom Serialization Example" section. The next part of the function, is where the list of items ans the luggages are loaded in the main part. 
\kode{Informs that there is saved to a file}{FileSerializer1}{FileSerializer1.txt}
Here it can be seen that the "File Serializer" needs some data descriptions to turn the information from the item list into a data file. 
This is done by stating a kind of sentence, where the first part is the name of the certain information, which needs to be saved into the file. 
\kode{Informs the user that data is loaded from a previously saved file.}{FileSerializer2}{FileSerializer2.txt}
For it to open the data file the program also needs to have data description on how the data is saved in the file. The program gets the information from the file, where the name of the certain information is, and puts it into the class list for each of the information that is saved.