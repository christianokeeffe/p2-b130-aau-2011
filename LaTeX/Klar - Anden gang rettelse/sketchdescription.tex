\section{Sketches before programming of GUI}
%Til retter: Vær OBS på første afsnit, er skrevet til
This section will describe the first thoughts of the GUI (Graphical User Interface). This is not how the final GUI is design, but was is the first sketches of the program.
Designing the program required some sketches and our first sketch looked like \ref{fig:skitse1}.
It shows some text boxes where the user must type the dimensions of the suitcase, so the program knows how much space it has to deal with, and that the measurement is the metric unit system.

\figur{1.0}{Skitse1.png}{Initial sketch of the program}{fig:skitse1}

It has been discussed whether there should be a 3D image or a 2D image of the suitcase, where the user can see where the items are placed. It was decided that the best way to show the packing, is by a rotatable 3D image, if it could be managed.

\figur{1.0}{Skitse3a.png}{Second sketch, now with added control buttons}{fig:skitse3a}
Figure \ref{fig:skitse3a} shows how our idea was before we began programming the program. The idea was that suitcase would be shown and then an item, one at a time. The user would then be able to see where each item should be in the suitcase according to program.


\figur{1.0}{Skitse3b.png}{A version of the program without a 3D viewer}{fig:skitse3b}
The sketch seen on figure \ref{fig:skitse3b} was made in case the 3D-viewer could not work.

On figure \ref{fig:skitse4}, we looked at how the packing list should look like, and what functions it should contain.

\figur{1.0}{Skitse4.png}{Sketch of the part where the user can add items}{fig:skitse4}

At last we decided upon the final sketch for the GUI. We replaced the 3D image viewer with the list of added items in the manage items and manage suitcases, because it gave a better overview of what the user had already added to the program. This can be seen on figure \ref{fig:skitse1done} 

\figur{1.0}{1.skitsedone.jpg}{Sketch of the main window}{fig:skitse1done}

What we changed from the first sketch of the main window is that more buttons was added. The two new buttons made it possible to load saved lists and start packing without needing to make changes to anything in the lists or making a new list.

\figur{1.0}{2.skitsedone.jpg}{Updated version of the window where the user can manage items}{fig:skitse2done}

Figure \ref{fig:skitse2done} is the sketch of the window that will open if the "Manage Items"-button was clicked. Here the 3D viewer has been replaced with a list of the items added in the program. The buttons are tools for adding, deleting or managing the items on the list while the "Save item list"-button lets you save the list so it can be used another time and returns you to the main window.

\figur{1.0}{3.skitsedone.jpg}{Window where the user can manage suitcases}{fig:skitse3done}

The window seen on figure \ref{fig:skitse3done} will open if the "Manage Suitcase"-button was clicked. Like in the "Manage Items" window the 3D viewer has been replaced by a list of all the added suitcases. Functionally does the buttons the same as those in the "Manage Items" window just with suitcases. Clicking the "Save Suitcase List"-button will let you save the list and will return you to the main window.

\figur{1.0}{4.skitsedone.jpg}{Sketch of the window that appear if the user clicks the "New packing list"-button}{fig:skitse4done}

Then clicking the "New packing list"-button the window seen on figure \ref{fig:skitse4done} will show. Here you are able to add the suitcases you want the program to pack in. On the left side in a window the program will show a 3D picture of the suitcase. The "Add items"-button will send you to the next window.

\figur{1.0}{5.skitsedone.jpg}{Window opened when the user adds new items}{fig:skitse5done}

In figure \ref{fig:skitse5done} all the things that the user wants to pack are needed to be added. After you have added all the items the "View list"-button will show a list of all the items and afterwards return you to the main window.

\figur{1.0}{6.skitsedone.jpg}{Sketch of a window that shows the saved lists}{fig:skitse6done}

Figure \ref{fig:skitse6done} is the window that will show if the "Load saved list"-button was clicked. Here the user would be able to load a saved list and delete list he/she does not want to save anymore. The "Back"-button will return you to the main window.

\figur{1.0}{7a.skitsedone.jpg}{Sketch of the 3D-viewer of the program}{fig:skitse7adone}

Figure \ref{fig:skitse7adone} is one of the suggestions on how the window showing where the items should be pack could look like. Here we have a 3D viewer showing the position of every item one item at a time. The two buttons are used to go forth and back through the items.

\figur{1.0}{7b.skitse.jpg}{Sketch of the 3D-viewer with instructions added}{fig:skitse7bdone}

Figure \ref{fig:skitse7bdone} is the second suggestion on how the window should look like. Here the 3D viewer is smaller but on the other hand has a list of instructions been added on the side to better help guide the user. The two buttons are used to go forth and back through the items.

\figur{1.0}{8.skitse.jpg}{Sketch of the message box that will show when the items has been packed}{fig:skitse8}

The message box seen on figure \ref{fig:skitse8} will show when all the items has been packed. The "Pack again"-button will show where to put all the things again. The "Make new list" will return the user to the main window where he/she can start making a new list, load a saved list or manage the already saved list. Most of the sketches do not look like the program what so ever, this is because the sketches where made before the program, and when the program was developed, the GUI was adapted to be more logical and from the experience from the tests.
